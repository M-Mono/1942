\documentclass[oneside,openright,headings=optiontohead]{scrbook}
\renewcommand{\baselinestretch}{1.3}  %行間距倍率
\columnsep 7mm
%\renewcommand\thepage{}


\usepackage[
a4paper=true,
%CJKbookmarks,
unicode=true,
bookmarksnumbered,
bookmarksopen,
hyperfigures=true,
hyperindex=true,
pdfpagelayout = SinglePage,
%pdfpagelayout = TwoPageRight,
pdfpagelabels = true,
pdfstartview = FitV,
colorlinks,
pdfborder=001,
linkcolor=black,
anchorcolor=black,
citecolor=black,
pdftitle={温故一九四二},
pdfauthor={刘震云},
pdfsubject={温故一九四二},
pdfkeywords={对于我来说,那是一个遥远的年代,遥远到如果不借助文字或者图片都无法去想象的年代。那一年,中国抗日战争进入战略相持阶段,抗日战争进入最重要的时期。自年一九三七抗日战争爆发,河南有几十万中国抗日军队驻防,而这几十万人的粮草补充,全靠自己省内解决。从一九三七年到一九四二年,五年半的时间,河南兵粮的贡献都是全国第一。沉重的兵役和赋税数额,使河南的民力物力财力枯竭,许多农民破产逃亡。其实就是在风调雨顺的时候,河南农民在交粮纳赋之后,也只能靠野菜和一些杂粮度日,更谈不上任何储藏。当时的百姓家都吃不上饭,许多百姓就被活活饿死。一九四二年河南全省遭灾,百姓的日子就更难过了。当时麦收只有一两成,秋粮甚至完全绝收,一场特大的饥荒就爆发了,这决不是偶然。颗粒无收的千百万民众不得不背井离乡,外出逃荒。},
pdfcreator={https://m-mono.github.io}
]{hyperref}


%Kindle Voyage, Paperwhite (3rd gen), Oasis: 6 in diagonal, 1448 × 1072 pixels @ 300 PPI 
\usepackage[papersize={9cm,12.2cm}]{geometry} % Kindle Format
\geometry{left=1.3cm,right=1.3cm,top=2cm,bottom=1cm,foot=4cm}
\usepackage{graphics,graphicx,pdfpages}


%自动加注拼音
\usepackage{xpinyin}
\xpinyinsetup{format={\color{PinYinColor}}}
%手动加注外语及日语振假名
\usepackage{ruby}
\renewcommand\rubysize{0.4} %匹配 xpinyin 默认标注字体大小
\renewcommand\rubysep{-0.5em} %匹配 xpinyin 默认标注高度

\usepackage{xeCJK}
\usepackage{indentfirst}
\setlength{\parindent}{2.0em}

%正文字体
\setCJKmainfont[Path=Fonts/]{WenYue-GuDianMingChaoTi-NC-W5.otf}
\setCJKsansfont[Path=Fonts/]{WenYue-GuDianMingChaoTi-NC-W5.otf}
\setCJKmonofont[Path=Fonts/]{WenYue-GuDianMingChaoTi-NC-W5.otf}
\setmainfont[Path=Fonts/]{WenYue-GuDianMingChaoTi-NC-W5.otf}
\setsansfont[Path=Fonts/]{WenYue-GuDianMingChaoTi-NC-W5.otf}
\setmonofont[Path=Fonts/]{WenYue-GuDianMingChaoTi-NC-W5.otf}
%补缺字体
\newfontfamily{\Add}[Path=Fonts/]{SourceHanSansCN-Normal.otf}



% 頁面及文字顏色
\usepackage{xcolor}
\definecolor{TEXTColor}{RGB}{50,50,50} % TEXT Color
\definecolor{PinYinColor}{RGB}{130,130,130} % TEXT Color
\definecolor{NOTEXTColor}{RGB}{0,0,0} % No TEXT Color
\definecolor{BGColor}{RGB}{240,240,240} % BG Color


\usepackage{multicol}

\makeindex
\renewcommand{\contentsname}{{温故一九四二}}
\usepackage{fancyhdr} % 設置頁眉頁腳
\pagestyle{fancy}
%\fancyhf{} % 清空當前設置
\renewcommand{\headrulewidth}{0pt}  %頁眉線寬,設為0可以去頁眉線
\renewcommand{\footrulewidth}{0pt}  %頁眉線寬,設為0可以去頁眉線

\usepackage{titletoc}
\dottedcontents{section}[100em]{\bfseries}{100em}{100em} % 去掉目录虚线


\begin{document}
	\frontmatter
	\begin{figure}[ht]
		\begin{center}
			\includepdf[height=\paperheight,width=\paperwidth]{Frontmatter.jpg}
		\end{center}
	\end{figure}
	\newpage
		\begin{center}
			\phantom {placeholder}
			\vspace{5mm}
			{\Huge 温故一九四二}\\
			\vspace{5mm}
			{\Huge 刘震云}
		\end{center}
	\newpage
	{\color{TEXTColor}
		\tableofcontents
		\newpage
		\mainmatter
			\include{./Chapters/01}
			\include{./Chapters/02}
			\include{./Chapters/03}
			\include{./Chapters/04}
			\include{./Chapters/05}
			\include{./Chapters/06}
			\fancyhead[LO]{{\scriptsize 【温故一九四二】附录}} %奇數頁眉的左邊
\fancyhead[RO]{\thepage} %奇數頁眉的右邊
\fancyhead[LE]{\thepage} %偶數頁眉的左邊
\fancyhead[RE]{{\scriptsize 【温故一九四二】附录}} %偶數頁眉的右邊
\fancyfoot[LE,RO]{}
\fancyfoot[LO,CE]{}
\fancyfoot[CO,RE]{}
\chapter*{附录}
\addcontentsline{toc}{chapter}{\hspace{11mm}附录}
%\thispagestyle{empty}
温故一九四二、一九四三年时,除了这场大灾荒,还有这些年代所发生的一些杂事。这些杂事中,最感兴趣的,是从当时的《河南明国日报》上,看到两则离异声明。这证明大灾荒只是当年的主旋律,主旋律之下,仍有百花齐放的正常复杂的情感纠纷和日常生活。我们不能以偏概全,一叶知秋,瞎子摸象,让巴掌山挡住眼。这就不全面了。我们不能只看到大灾荒,看不到人的全貌。从这一点说,我们对委员长的指责,也有些偏激了。另外,我们从这两则离异声明中,也可以看到时代的进步。下边是全文:\\

\begin{quote}
	紧要启事\\

缘鄙人与冯氏结婚以来感情不和难以偕老经双方同意自即日起业已离异从此男婚女嫁\\

各听自便此启\\

张荫萍冯氏启\\

声明启示\\

鄙人旧历十二月初六日赴洛阳送货鄙妻刘化许昌人该晚逃走将衣服被褥零碎物件完全带走至今数日音信全无如此人在外发生意外不明之事与鄙人无干自此以后脱离夫妻关系恐亲友不明特此登报郑重声明偃师槐庙村中正西街门牌五号田光寅启\\

\begin{flushright}
	一九九三年十二月北京十里堡\\
\end{flushright}
\end{quote}

\begin{center}
	- 全书完 -
\end{center}
		\backmatter
		{\color{TEXTColor}
			\begin{figure}[ht]
				\begin{center}
					\includepdf[height=\paperheight]{Backmatter.jpg}
				\end{center}
			\end{figure}
\end{document}