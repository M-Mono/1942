\fancyhead[LO]{{\scriptsize 【温故一九四二】第五章}} %奇數頁眉的左邊
\fancyhead[RO]{\thepage} %奇數頁眉的右邊
\fancyhead[LE]{\thepage} %偶數頁眉的左邊
\fancyhead[RE]{{\scriptsize 【温故一九四二】第五章}} %偶數頁眉的右邊
\fancyfoot[LE,RO]{}
\fancyfoot[LO,CE]{}
\fancyfoot[CO,RE]{}
\chapter*{五}
\addcontentsline{toc}{chapter}{\hspace{11mm}第五章}
%\thispagestyle{empty}
河南开始救灾。因为委员长动作了。委员长说要救灾,当然就救灾了。不过,在一九四二、一九四三年,首起救灾民于水深火热之中的,仍然是外国人。虽然我们讨厌外国人,不想总感谢他们,但一到关键时候,他们还真来帮我们,让我们怎么办呢?这时救灾的概念,已不是整体的、宏观的、从精神到物质的,仅仅是能填一下快饿死过去人的肚子,把人从生命死亡线上往回拉一把。外国主教们{\Add ──}本来是来对我们进行精神侵略{\Add ──}在委员长动作之前,已经开始自我行动了。这个行动不牵涉任何政治动机,不包含任何政府旨意,而纯粹是从宗教教义出发。他们是受基督委派前来中国传教的牧师,干的是慈善事业。这里有美国人,也有欧洲人;有天主教徒,也有新教徒。尽管美国人和意大利人正在欧洲互相残食,但他们的神父在我的故乡却携手共进,共同从事着慈善事业,在尽力救着我多得不可数计的乡亲的命。人在战场上是对立的,但在我一批批倒下的乡亲面前,他们的心却相通了。从这一点上说,我的乡亲们也不能说饿死得全无价值。教会一般是设粥场;而有教会的地方,一般在城市如郑州、洛阳等。我的几个亲戚,如二姥娘一家、三姥娘一家,都喝过美国、欧洲人在大锅里熬制的粥。我的花爪舅舅,就是在洛阳到粥场领粥的路上,被胡宗南将军抓了壮丁的。慈善机构从哪里来的粮食熬粥呢?因为美国政府对蒋也不信任了,外来的救济物质都是通过传教士实行发放的;而这些逃窜的中国灾民,虽然大字不识,但也从本能出发,对本国政府失去信任,感到唯一的救星就是外国人、白人。白修德记载:\\

教士们只是在必要时才离开他们的院子。因为唯有在大街上走着的一个白人才能给难民们带来希望。他会突然被消瘦的男子、虚弱的妇女和儿童围住。他们跪在地上,匍匐着,磕着头,同时凄声呼喊:“可怜可怜吧!”但他们恳求的实际上不过是一点食物。\\

读到这里,我一点不为我的乡亲脸红。如果换了我,处在当时那样的处境,我也宁愿给洋人磕头。教会院子周围,到处是逃难的人群。传教士一出院子,就被围得水泄不通。乡亲们都聚集到外国人周围了。我想这时如外国人振臂一呼,乡亲们肯定会跟他们揭竿而起,奋勇前进,视死如归,再不会发生八国联军时抵抗外国人的情形了。儿童和妇女们,每日坐在教会门口;每天早晨,传教士们必须把遗弃在教会门前的婴儿送进临时设立的孤儿院去抚养{\Add ──}连后代也托付给洋人了。唯有这些少数外国人,才使我的乡亲意识到生命是可贵的。我从发黄的五十年前的报纸上看到,一个外国天主教神父在谈到设立粥场的动机时说:\\

至少要让他们像人一样死去。\\

教会还开办了教会医院。教会医院里挤满了可怕的肠胃病患者。疾病的起因是:他们都食用了污秽不堪的东西。许多难民在饥饿难当时,都拼命把泥土塞进嘴里,以此来装填他们的肚子。医院要救活这些人,必须首先想办法把泥土从这些人的肚子里掏出来。\\

教会还设立了孤儿院,用来收留父母饿死后留下的孩子。但这收留必须是秘密的。因为如大张旗鼓说要收留孩子,那天下的孤儿太多了;有些父母不死的,也把自己的孩子丢弃或倒卖了。外国人太少,中国孤儿太多;换言之,中国孩子想认外国人做爹的太多,外国人做爹也做不过来。一个资料这样记载:\\

饥饿甚至毁灭了人类最起码的感情:一对疯狂的夫妇,为了不让孩子们跟他们一起出去,在他们外出寻找食物时,把他们的六个孩子全都捆绑在树上;一位母亲带着一个婴儿和两个大一点的孩子外出讨饭,艰难的长途跋涉使他们非常疲倦,母亲坐在地上照料婴儿,叫两个大一些的孩子再走一个村子去寻找食物,等到两个孩子回来,母亲已经死了,婴儿却还在吸吮着死人的乳头;有一对父母杀死了他们的两个孩子,因为他们宁愿这样做也不愿再听到孩子乞求食物的哭叫声。传教士们尽力沿途收捡弃儿,但他们必须偷偷地做,因为这消息一经传扬出去,立即就会有无数孩子被丢弃在他们的门口,使他们无法招架。\\

儿童是一个国家或一个政府的晴雨表。就像如果儿童的书包过重、人为规定的作业带到家里还做不完压得儿童喘不过气,证明这个国家步履蹒跚一样,如果一个政府在儿童一批批饿死它也听任不管而推给外国人的话,这个政府到底还能存在多长时间,就值得怀疑了。连外国人都认为,如果身体健康,中国的儿童是非常漂亮的,他们的头发有着非常好看的自然光泽,他们那杏仁一样的眼珠闪动着机灵的光芒。但是,现在这些干瘦、萎缩得就像稻草人似的孩子,在长眼睛的地方却只有两个充满了脓液的裂口,饥饿使得他们腹部肿胀,寒冷干燥的气候使得他们的皮肤干裂,他们的声音枯竭,只能发出乞讨食物的微弱哀鸣。{\Add ──}这只代表儿童本身吗?不,也代表着国民政府。如果坐在黄山别墅的蒋委员长,是坐在这样一群儿童的国民头上,他的自信心难道不受影响吗?他到罗斯福和丘吉尔面前,罗、邱能够看得起他吗?\\

毕竟,蒋还是人{\Add ──}说道谁还是个人这句话,每当我听到这句话,譬如,一个妻子说丈夫或丈夫说妻子:“你也算个人!”我心里就感到莫大的悲哀。这是多么轻蔑的话语!这是世界的末日!但蒋还是个人,当外国记者把一张狗吃人的照片摆在他面前时(多么小的动因),他毕竟也要在外国人之后关心我故乡三千万灾民了。他在一批人头落地后,也要救灾了。即:中国也要救灾了。但中国的救灾与外国人的救灾也有不同。外国人救灾是出于作为人的同情心、基督教义,不是罗斯福、丘吉尔、墨索里尼发怒后发的命令;中国没有同情心,没有宗教教义,(蒋为什么信基督教呢?纯粹为了结婚和性交或政治联姻吗?)有的只是蒋的一个命令。{\Add ──}这是中西方的又一区别。\\

那么中国政府又是怎么救灾的呢?我再引用几段资料。也许读者对我不厌其烦地引征资料已经厌烦了,但没有办法,为了保持历史的真实性,就必须这么做,烦也没办法,烦也不是我的责任,这不是写小说,这是朋友交给我的任务与我日常任务的最大区别。我也不想引用资料,资料束缚得我毫无自由,如缚着绳索。但我的朋友给我送了一大捆资料。我当时有些发怵:\\

“得看这么多资料吗?”\\

朋友:\\

“为了防止你信马由僵和瞎编!”\\

所以,我只好引用这些资料。至于这些资料因为朋友的原因过多地出现在我的文字里,请大家因为我暗含委屈而能够原谅我。\\

中国政府在一九四三年救灾的资料:\\

\begin{quote}
	\begin{description}
		\item {\Add Δ} 委员长下达了救灾的命令。\\
		
		\item {\Add Δ} 但是,愚蠢和效率低下是救济工作的特点。由于各地地方官员的行为恶劣,可怕的悲剧甚至进一步恶化。\\
		
		\item {\Add Δ} 本来,陕西省与河南省相毗邻,陕西的粮食储存较为丰富,作为一个强有力的政府,就应该下令立刻把粮食从陕西运到河南以避免灾祸。然而,这样一来便有利于河南而损害了陕西,就会破坏政府认为必不可少的微妙的权势平衡,而政府是不会答应的。 (中国历来政治高于人,政治是谁创造的呢?创造政治为了什么呢?)此外,还可以从湖北运送粮食到河南,但是湖北的战区司令长官不允许这样做。\\
		
		\item {\Add Δ} 救济款送到河南的速度很慢。(纸币有什么用,当那里再无食物可以购买的话,款能吃吗?)经过几个月,中央政府拨给的两亿元救济款中只有八千万元运到了这里。甚至这些已经运到的钱也没有发挥出救灾作用。政府官员们把这笔钱存入省银行,让它生利息;同时又为怎样最有效地使用这笔钱争吵不休。在一些地区,救济款分配给了闹饥荒的村庄。地方官员收到救济款后,从中扣除农民所欠的税款,农民实际能得到的没有多少。就连国家银行也从中渔利。中央政府拨出的救济款都是面额为一百元的钞票。这样的票面已经够小的了,因为每磅小麦售价达十元至十八元。但是,当时的粮食囤积者拒绝人们以百元票面的钞票购买粮食。要购买粮食的农民不得不把这钞票兑换成五元和十元的钞票,这就必须去中央银行。国家银行在兑换时大打折扣,大钞票兑换小钞要抽取百分之十七的手续费。河南人民所需要的是粮食,然而直到三月份为止,政府只供应了大约一万袋大米和两万袋杂粮。从秋天起一直在挨饿的三千万河南人民,平均每人大约只有一磅粮食。\\
		
		\item {\Add Δ} (救灾之时),农民们仍处在死亡之中,他们死在大路上、死在山区里、死在火车站旁、死在自己的泥棚内、死在荒芜的田野中。\\
	\end{description}
\end{quote}

当然,并不是所有的政府官员都这么黑心烂肺,看着人民死亡还在盘剥人民。也有良心发现,想为人民办些好事或者想为自己树碑立传的人。我历来认为,作为我们这些普通百姓,只要能为我们办些或大或小的好事,官员的动机我们是不追究的,仅是为了为人民服务也好,或是为了创造政绩升官也好,或是为了向某个情人证明什么也好,我们都不管,只要为我们做好事。仁慈心肠的汤恩伯将军就在这时站了出来,步洋人的后尘,学洋人的洋子,开办了一个孤儿院,用来收留洋人收剩余的孤儿。这是好事。汤将军是好人。但这是一个什么样的孤儿院呢?白修德写道:\\

在我的记忆中,中央政府汤恩伯将军办的孤儿院是一个臭气熏天的地方。连陪同我们参观的军官也受不了这种恶臭,只好抱歉地掏出手绢捂住鼻子,请原谅。孤儿院所收容的都是被丢弃的婴儿,四个一起放在摇篮里。放不进摇篮的干脆就放在稻草上。我记不得他们吃些什么了。但是他们身上散发着呕吐出来的污物和屎尿的臭气。孩子死了,就抬出去埋掉。\\

就是这样,我们仍说汤将军好。因为汤将军已是许多政府官员和将军中最好的了。就是这样的孤儿院,也比没有孤儿院要好哇。\\

还有的好人在进行募捐和义演。所谓募捐和义演,就是在民间募捐,由演员义演,募得义演的钱,交给政府,由政府再去发放给灾民。一九四二年的《河南民国日报》,在十一月份的报纸上,充斥了救灾义演、救灾音乐会、书画义卖、某某捐款的报道。我所在家乡县的县政府韩书记,就曾主持过一场义演。我相信,参加募捐和义演的人,心都是诚的,血都是热的,血浓于水,流下不少同情我们的眼泪。但问题是,募捐和义演所得,并不能直接交到我们手中,而是要有组织地交给政府,由政府再有组织地分发给灾民。这样,中间就经过许多道政府机构{\Add ──}由省到县,由县到乡,由乡到村{\Add ──}的中间环节。这么多道中间环节,就使我们很不放心了。中央政府的救济款,还层层盘剥,放到银行生利息,到了手中又让大票兑小票,收取百分之十七的手续费;这募捐和几个演员赚得的钱,当经过他们手时,能安全迅速通达到我们这里吗?我们不放心哩。\\

这些就不说了。政府是爹娘,打骂克扣我们,就如同打掉我们的牙我们可以咽下;问题严重还在于,我们民间一些志人志怪、有特殊才能的人,这时也站了出来。不过不是站到我们灾民一边{\Add ──}站在我们一边对他有什么用呢?而是站在政府一边,替政府研究对付饥饿的办法。如《河南民国日报》一九四三年二月十四日载:\\

财政科员刘道基,目前已发明配制出救荒食品,复杂的吃一次七天不饿,简易的吃一次一天不饿。\\

任何一个中国人,五十年后,在读到这条简短消息时,我想情感都是很复杂的。看来不但政府依靠不得,连一个科员,我们自己的下层兄弟,也指望不得了。如这种发明是真实的,可行的,当然好;政府欢迎,不用再救灾;我们也欢迎,不用再死人。不但当时的政府欢迎,在以后几十年的中国历史上,饿死人的事也是不断发生的,如有这种人工配制吃一次七天不饿的东西,中国千秋万代可保太平。但这种配制没有流传到今天,可见当时它也只是起了宣传作用、稳定人心作用,并没有救活我们一个人。也许刘道基先生是出于好心、同情心、耐心和细心,也许想借此升官,但不管他个人出于什么动机,这配制也对我们无用。我们照常一天一天在饿死,死在大路上、田野中和火车站旁。\\

{\Add ──}这就是一九四三年在蒋介石先生领导下的救灾运动。如果用总结性的话说,这是一场闹剧,一场只起宣传作用或者只是做给世界看做给大家看做给洋人洋人政府看的一出闹剧。委员长下令求灾,但并无救灾之心,他心里仍在考虑世界和国家大事,各种政治势力的平衡。这是出演闹剧的症结。闹剧中的角色林林总总,闹剧的承受者仍是我们灾民。这使我不禁想起了毛泽东的一句话:问苍茫大地,谁主沉浮?我说:我们死不死,有谁来管?作为我们即将死去的灾民,态度又是如何呢?《大公报》记者张高峰记载:\\

河南人是好汉子,眼看自己要饿死,还放出豪语来:“早死晚不死,早死早脱生!”\\

娘啊,多么伟大的字眼!谁说我们的民族没有宗教?谁说我们的民族没有向心力,是一盘散沙?我想就是佛祖面临这种情况,也不过说出这句话了。委员长为什么信基督呢?基督教帮过你什么?就帮助你找了一个老婆;而深入中国人灵魂深处的佛家教义,却在一九四二至一九四三年,帮了你政治的大忙。\\

当然,在这场灾难中,三千万河南人,并不是全饿死了,死的还是少数:三百万。十分之一。逃荒逃了三百万。剩下的河南人还有两千多万。这不死的两千多万人,在指望什么呢?政府指望不得,人指望不得,只有盼望大旱后的土地,当然,土地上也充满了苛捐杂税和压榨。但这毕竟是唯一可以指望的东西。据记载,大旱过后的一九四三年冬天(指年初的冬天),河南下了大雪;七月份又下了大雨。这是好兆头。我们盼望在老天的关照下,夏秋两季能有一个好收成。只要有了可以裹腹的粮食,一切都好说,哪怕是一个充满黑暗、丑恶、污秽和盘剥的政府,我们也可以容忍。我们相信,当时的国民政府,在这一点上,倒能与我们心心相通,希望老天开眼,大灾过去,风调雨顺,能有一个好收成。不然情况继续下去,把人一批批全饿死了,政府建在哪里呢?谁给政府中的首脑和各级官员提供温暖的住处和可口的食物然后由他们的头脑去想对付百姓的制度和办法也就是政治呢?人都没有了,它又去统治谁呢?但老天没有买从政府到民众两千多万人的帐,一九四三年祸不单行,大旱之后,又来了蝗灾。这更使我们这些灾民的命运雪上加霜。\\