\fancyhead[LO]{{\scriptsize 【温故一九四二】第六章}} %奇數頁眉的左邊
\fancyhead[RO]{\thepage} %奇數頁眉的右邊
\fancyhead[LE]{\thepage} %偶數頁眉的左邊
\fancyhead[RE]{{\scriptsize 【温故一九四二】第六章}} %偶數頁眉的右邊
\fancyfoot[LE,RO]{}
\fancyfoot[LO,CE]{}
\fancyfoot[CO,RE]{}
\chapter*{六}
\addcontentsline{toc}{chapter}{\hspace{11mm}第六章}
%\thispagestyle{empty}
蝗灾发生于一九四三年秋天,关于蝗灾的描写,我知道主编《百年灾害史》的朋友另有安排,我这篇《温故一九四二》,重点不在蝗虫。关于蝗虫,中国历史上有更大规模的阵仗;另一位我所敬重的朋友,正在描写它们。但这并不影响我对它们的提及,因为我们分别描写的是不同年代的蝗虫。他写的是一九二七年的山东的蝗虫,我写的是一九四二年生活在我故乡的蝗虫,蝗灾相似,蝗虫不同。据俺姥娘说,一九四三年的蝗虫个大,有绿色的(我想是年轻的),有黄色的(我想是长辈),成群结队,遮天蔽日,像后来发生的太平洋战争或诺曼底登陆时的轰炸机机群一样,老远就听到“嗡嗡”的声音,说俯冲,大家都俯冲,覆盖了一块庄稼地;一个时辰,这块庄稼地就没有了。一九四三年的春天,风打麦,颗粒无收;秋天又遇到蝗虫,灾民的生活,就可想而知了。蝗虫来了,人死了,正在继续一批一批地死去。据俺爹俺姥娘讲,蝗虫不吃绿豆,不吃红薯,不吃花生,不吃豇豆,吃豆子、玉米、高粱。为了维护自己的生命,我故乡还无死光的难民,与蝗虫展开了大战。政府我们没办法,它的盘剥和压榨往往通过一架疯狂运转的机器,何况他们有枪;但蝗虫我们可以面对面地与它作战,且没有谋反暴动的嫌疑。这是蝗虫与政府的区别。\\

怎么搏斗?三种办法:\\

\begin{quote}
		\begin{description}
	
		\item [一、] 把床单子绑在竹竿上挥舞,驱赶蚂蚱。但这是损人利己的做法,你把蚂蚱赶走,蚂蚱不在你这块田里,就跑到了别人的田里;何况你今天赶走,明天就又来了。\\
		
		\item [二、] 田与田之间挖大沟,阻挡蚂蚱的前进。蚂蚱吃完这块地,向另一块转移时,要经过大沟,这时就用舂米的碓子砸蚂蚱,把它们砸成烂泥;或用火烧;这种做法有些残忍,但消灭蝗虫较彻底;我想被乡亲们杵死的蚂蚱,也一定像当年饿死的乡亲一样多。\\
		
		\item [三、] 求神。我姥娘就到牛进宝的姑姑所设的香坛去烧过香,求神保护她的东家的土地不受蚂蚱的侵害。但据资料表明,乡亲们所做的这一切,都是白费。蚂蚱太多,靠布单子,靠沟,靠神,都没有解决问题,蝗虫照样吃了他们的大部分庄稼。灾民在一九四二年是灾民,到一九四三年仍是灾民。\\
	\end{description}
\end{quote}

自然的暴君,又开始摇撼河南农民的生命线,旱灾烧死了他们的麦子,蝗虫吃了他们的高粱,冰雹打死了他们的荞麦,最后的希望又随着一棵棵垂毙的秋苗枯焦,把他们赶上死亡的路途。那时的河南人,十之八九困于饥饿中。\\

照此下去,我想我故乡的河南人,总有一天会被饿死光。这是我们和我们的政府不愿意看到的。后来事实证明,河南人没有全部被饿死,很多人还流传下来,繁衍生息,五十年后,俨然又是在人口上的中国第二大省。当时为什么没有死绝呢?是政府又采取什么措施了吗?不是。是蝗虫又自动飞走了吧?不是。那是什么?是日本人来了{\Add ──}一九四三年,日本人开进了河南灾区,这救了我的乡亲们的命。日本人在中国犯了滔天罪行,杀人如麻,血流成河,我们与他们不共戴天;但在一九四三年冬至一九四四年春的河南灾区,却是这些杀人如麻的侵略者,救了我不少乡亲的命。他们给我们发放了不少军粮。我们吃了皇军的军粮,生命得以维持和壮大。当然,日本发军粮的动机绝对是坏的,心不是好心,有战略意图,有政治阴谋,为了收买民心,为了占我们的土地,沦落我们河山,奸淫我们的妻女,但他们救了我们的命;话说回来,我们自己的政府,对待我们的灾民,就没有战略意图和政治阴谋吗?他们对我们撒手不管。在这种情况下,为了生存,有奶就是娘,吃了日本的粮,是卖国,是汉奸,这个国又有什么不可以卖的呢?有什么可以留恋的呢?你们为了同日军作战,为了同共产党作战,为了同盟国,为了东南亚战争,为了史迪威,对我们横征暴敛,我们回过头就支持日军,支持侵略者侵略我们。所以,当时我的乡亲们,我的亲戚朋友,为日军带路的,给日军支前的,抬担架的,甚至加入队伍、帮助日军去解除中国军队武装的,不计其数。五十年后,就是追查汉奸,汉奸那么多,遍地都是,我们都是汉奸的后代,你如何追查呢?据资料记载,在河南战役的几个星期中,大约有五万名中国士兵被自己的同胞缴了械。我们完整地看一下资料:\\

一九四四年春天,日军决定在河南省进行大扫荡,以此为他们在南方进行一次更大规模的攻势作准备。河南战区名义上的司令官是一位目光炯炯的人物,名叫蒋鼎文。在河南省内,他最拿手的好戏是在他的辖区内恐吓行政官员。他曾责骂河南省主席,使这位主席在恐慌之中与他合作制定了一个计划,这个计划剥夺了农民手中最后一点粮食。\\

日军进攻河南时使用的兵力大约为六万人。日军于四月中旬发起攻击,势如破竹地突破了中国军队的防线。这些在灾荒之年蹂躏糟蹋农民的中国军队,由于多年的懒散,它本身也处于病态,而且士气非常低落。由于前线的需要,也是为了军官们自己的私利,军队开始强行征用农民的耕牛以补充运输工具。河南是小麦种植区,耕牛是农民的主要生产资料,强行征牛是农民不堪忍受的。\\

农民们一直等待着这个时机。连续几个月以来,他们在灾荒和军队残忍的敲诈勒索之下,忍着痛苦的折磨。现在,他们不再忍受了。他们用猎枪、大刀和铁耙把自己武装起来。开始时他们只是缴单个士兵的武器,最后发展到整连整连地解除军队的武装。据估计,在河南战役的几个星期中,大约有五万名中国士兵被自己的同胞缴械了。在这种情况下,如果中国军队能维持三个月,那真是不可思议的事情。整个农村处于武装暴动的状态,抵抗毫无希望。三个星期内,日军就占领了他们的全部目标,通往南方的铁路也落入日军之手,三十万中国军队被歼灭了。\\

日本为什么用六万军队,就可以一举歼灭三十万中国军队?在于他们发放军粮,依靠了民众。民众是广大而存在的。一九四三年至一九四四年春,我们就是帮助了日本侵略者。汉奸乎?人民乎?白修德在战役之前采访一位中国军官,指责他们横征暴敛时,这位军官说:“老百姓死了,土地还是中国人的;可是如果当兵的饿死了,日本人就会接管这个国家。”这话我想对委员长的心思。当这问题摆在我们这些行将饿死的灾民面前时,问题就变成:是宁肯饿死当中国鬼呢?还是不饿死当亡国奴呢?我们选择了后者。\\

这是我温故一九四二,所得到的最后结论。\\