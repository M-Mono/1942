\fancyhead[LO]{{\scriptsize 【温故一九四二】第一章}} %奇數頁眉的左邊
\fancyhead[RO]{\thepage} %奇數頁眉的右邊
\fancyhead[LE]{\thepage} %偶數頁眉的左邊
\fancyhead[RE]{{\scriptsize 【温故一九四二】第一章}} %偶數頁眉的右邊
\fancyfoot[LE,RO]{}
\fancyfoot[LO,CE]{}
\fancyfoot[CO,RE]{}
\chapter*{一}
\addcontentsline{toc}{chapter}{\hspace{11mm}第一章}
%\thispagestyle{empty}
我姥娘将五十年前饿死人的大旱灾,已经忘得一乾二净。我说:\\

“姥娘,五十年前,大旱,饿死许多人!”\\

姥娘:\\

“饿死人的年头多得很,到底指的哪一年?”\\

我姥娘今年九十二岁。与这个世纪同命运。这位普通的中国乡村妇女,解放前是地主的雇工,解放后是人民公社社员。在她身上,已经承受了九十二年的中国历史。没有千千万万这些普通的肮脏的中国百姓,波澜壮阔的中国革命和共产党历史都是白扯。他们是最终的灾难和成功的承受者和付出者。但历史历来与他们无缘,历史只漫步在富丽堂皇的大厅。所以俺姥娘忘记历史一点没有惭愧的脸色。不过这次旱灾饿死的是我们身边父老乡亲,是自己人,姥娘的忘记还是稍稍有些不对。姥娘是我的救命恩人。这牵涉到另一场中国灾难{\Add ──}一九六零年。老人家性情温和,虽不识字,却深明大义。我总觉中国所以能发展到今天,仍给人以信心,是因为有这些性情温和、深明大义的人的存在而不是那些心怀叵测、并不善良的人的生存。值得我欣慰的是,仗着一位乡村医生,现在姥娘身体很好,记忆力健全,我母亲及我及我弟弟妹妹小时候的一举一动,仍完整地保存在她的记忆里。我相信她对一九四二年的忘却,并不是一九四二年不触目惊心,而是在老人家的历史上,死人的事确是发生得太频繁了。指责九十二年许许多多的执政者毫无用处,但在哪位先生的执政下他的黎民百姓经常、到处被活活饿死,这位先生确应比我姥娘更感到惭愧。这个理应惭愧的前提是;他的家族和子孙,决没有发生饥饿。当我们被这样的人统治着时,我们不也感到不放心和感到后怕吗?但姥娘平淡无奇的语调,也使我的激动和愤怒平淡起来,露出自嘲的微笑。历史从来是大而化之的。历史总是被筛选和被遗忘的。谁是执掌筛选粗眼大筐的人呢?最后我提起了蝗虫。一九四二年的大旱之后,发生了遮天蔽日的蝗虫。这一特定的标志,勾起了姥娘并没忘却的蝗虫与死人的联系。她马上说:\\

“这我知道了。原来是飞蚂蚱那一年。那一年死人不少。蚂蚱把地里的庄稼都吃光了。牛进宝他姑姑,在大油坊设香坛,我还到那里烧过香!”\\

我说:\\

“蚂蚱前头,是不是大旱?”\\

她点着头:\\

“是大旱,是大旱,不大旱还出不了蚂蚱。”\\

我问:\\

“是不是死了很多人?”\\

她想了想:\\

“有个几十口吧。”\\

这就对了。一个村几十口,全省算起来,也就三百万了。我问:\\

“没死的呢?”\\

姥娘:\\

“还不是逃荒。你二姥娘一股人,三姥娘一股人,都去山西逃荒了。”\\

现在我二姥娘、三姥娘早已经不在了。二姥娘死时我依稀记得,一个黑漆棺材;三姥娘死时我已二十多岁,记得是一颗苍白的头,眼瞎了,像狗一样蜷缩在灶房的草铺上。他的儿子我该叫花爪舅舅的,在村里当过二十四年支书,从一九四八年当到一九七二年,竟没有治下一座象样的房子,被村里人嘲笑不已。放下二姥娘三姥娘我问:\\

“姥娘,你呢?”\\

姥娘:\\

“我没有逃荒。东家对我好,我又去给东家种地了。”\\

我:\\

“那年旱得厉害吗?”\\

姥娘比着:\\

“怎么不厉害,地裂得像小孩子嘴。往地上浇一瓢水,‘滋滋’冒烟。”\\

这就是了。核对过姥娘,我又去找花爪舅舅。花爪舅舅到底当过支书,大事清楚,我一问一九四二年,他马上说:\\

“四二年大旱!”\\

我:\\

“旱成甚样?”\\

他吸着我的“阿诗玛”烟说:\\

“一入春就没下过雨,麦收不足三成,有的地块颗粒无收;秧苗下种后,成活不多,活的也长尺把高,结不成籽。”\\

我:\\

“饿死人了吗?”\\

他点头:\\

“饿死几十口。”\\

我:\\

“不是麦收还有三成吗?怎么就让饿死了?”\\

他瞪着我:\\

“那你不交租子了?不交军粮了?不交税赋了?卖了田也不够纳粮,不饿死也得让县衙门打死!”\\

我明白了。我问:\\

“你当时有多大?”\\

他眨眨眼:\\

“也就十五六岁吧。”\\

我:\\

“当时你干什么去了?”\\

他:\\

“怕饿死,随俺娘到山西逃荒去了。”\\

撇下花爪舅舅,我又去找范克俭舅舅。一九四二年,范克俭舅舅家在我们当地是首屈一指的大户人家。我姥爷姥娘就是在他家扛的长工。东家与长工,过从甚密;范克俭舅舅几个月时,便认我姥娘为干娘。俺姥娘说,一到吃饭时候,范克俭他娘就把范克俭交给我姥娘,俺姥娘就把他放到裤腰里。一九四九年以后,主子长工的身份为之一变。俺姥娘家成了贫农,范克俭舅舅的爹在镇反中让枪毙了,范克俭舅舅成了地主分子,一直被管制到一九七八年。他的妻子、我的金银花舅母曾向我抱怨,说她嫁到范家一天福没享,就跟着受了几十年罪,图个啥呢?因为她与范克俭舅舅结婚于一九四八年底。但在几十年中,我家与范家仍过从甚密。范克俭舅舅见了俺姥娘就“娘、娘”地喊。我亲眼见俺姥娘拿一块月饼,像过去的东家对她一样,大度地将月饼赏给叫“娘”的范克俭舅舅。范克俭舅舅脸上露出感激的笑容。我与范克俭舅舅,坐在他家院中一棵枯死的大槐树下(这颗槐树,怕是一九四二年就存在吧?)共同回忆一九四二年。一开始范克俭舅舅不知一九四二年为何物,“一九四二年?一九四二年是哪一年?”这时我想起他是前朝贵族,不该提四九年以后实行的公元制,便说是民国三十一年。谁知不提民国三十一年还好些,一提民国三十一年范克俭舅舅暴跳如雷:\\

“别提民国三十一年,三十一年坏得很。”\\

我吃惊:\\

“三十一年为什么坏?”\\

范克俭舅舅:\\

“三十一年俺家烧了一座小楼!”\\

我不明白:\\

“为什么三十一年烧小楼?”\\

范克俭舅舅:\\

“三十一年不是大旱吗?”\\

我答:\\

“是呀,是大旱!”\\

范克俭舅舅:\\

“大旱后起蚂蚱!”\\

我:\\

“是起了蚂蚱!”\\

范克俭舅舅:\\

“饿死许多人!”\\

我:\\

“是饿死许多人!”\\

范克俭舅舅将手中的“阿诗玛”烟扔了一丈多远:\\

“饿死许多人,剩下没饿死的穷小子就滋了事。挑头的是毋得安,拿着几把大铡、红缨枪,占了俺家一座小楼,杀猪宰羊,说要起兵,一时来俺家吃白饭的有上千人!”\\

我为穷人辩护:\\

“他们也是饿得没办法!”\\

范克俭舅舅:\\

“饿得没办法,也不能抢明火呀!”\\

我点头:\\

“抢明火也不对,后来呢?”\\

范克俭舅舅诡秘地一笑:\\

“后来,后来小楼起了大火,麻杆浸着油。毋得安一帮子都活活烧死了,其它就做鸟兽散!”\\

“唔。”\\

是这样。大旱。大饥。饿死人。盗贼蜂起。\\

与范克俭舅舅分手,我又与县政协委员、四九年之前的县书记坐在一起。这是一个高大的、衰败的、患有不住摆头症的老头。虽然是县政协委员,但衣服破旧,上衣前襟上到处是饭点和一片一片的油渍。虽是四合院,但房子破旧,瓦檐上长满了枯黄的杂草。还没问一九四二年,他对他目前的境况发了一通牢骚。不过我并不觉得这牢骚多么有理,因为他的鼎盛时期,是四九年之前当县书记的时候。不过那时的县书记,不能等同于现在的县委书记,现在的县委书记是全县上百万人的父母官,那时的县书记只是县长的一个笔录,何况那时全县仅二十多万人。不过当我问起一九四二年,他马上不发牢骚了,立即回到了年轻力壮的鼎盛时期,眼里发出光彩,头竟然也不摇了。说:\\

“那时方圆几个县,我是最年轻的书记,仅仅十八岁!”\\

我点头。说:\\

“韩老,据说四二年大旱很厉害?”\\

他坚持不摇头说:\\

“是的,当时有一场常香玉的赈灾义演,就是我主持的。”\\

我点头。对他佩服。因为在一九九一年,中国南方发水灾,我从电视上见过赈灾义演。我总觉得把那么多鱼龙混杂的演艺人集合在一起,不是件容易的事。没想到当年的赈灾义演,竟是他主持的。接着老人家开始叙述当时的义演盛况及他的种种临时抱佛脚的解决办法。边说边发出爽朗开心的笑声。等他说完,笑完,我问:\\

“当时旱象如何?”\\

他:\\

“旱当然旱,不旱能义演?”\\

我绕过义演,问:\\

“听说饿死不少人,咱县有多少人?”\\

他开始摇头,左右频繁而有节奏地摇摆。摆了半天说:\\

“总有个几万人吧。”\\

看来他也记不清了。几万人对于当时的笔录书记,似也没有深刻的记忆。我告别他及义演,不禁长出一口气,也像他一样摇起头来。\\

这是在我故乡河南延津县所进行的旱情采访。据河南省志载,延津也是当时旱灾最严重的县份之一。但我这些采访都是零碎的,不完全、不准确的,五十年后,肯定夹杂了许多当事人的记忆错乱和本能的按个人兴趣的添枝或减叶。这不必认真。需要认真的,是当时《大公报》重庆版驻河南的战场记者高峰的一篇报道。这篇报道采访于当年,发表于当年,真实可靠性起码比我的同乡更真实可靠一些。这篇报道的标题是:《豫灾实录》。里边不但描写了旱灾与饥饿,还写到饥饿的人们在灾难里吃的是什么。这使我深深体会到,翻阅陈旧的报纸比到民间采访陈旧的年头便当多了。我既能远离灾难,又能吃饱穿暖居高临下地对灾难中的乡亲给予同情。\\

这篇报道写于一九四三年一月十七日。\\

\begin{quote}
	\begin{description}
		\item [$\bigtriangleup$] 记者首先告诉读者,今日的河南已有成千成万的人正以树皮(树叶吃光了)与野草维持着那可怜的生命。“兵役第一”的光荣再没有人提起,“哀鸿遍野”不过是吃饱穿暖了的人们形容豫灾的凄楚字眼。\\

		\item [$\bigtriangleup$] 河南今年(指旧历,乃是一九四二年)大旱,已用不着我再说。“救济豫灾”这伟大的同情,不但中国报纸,就是同盟国家的报纸也印上了大字标题。我曾为这四个字“欣慰”,三千万同胞也引颈翘望,绝望了的眼睛又发出了希望的光。希望究竟是希望,时间久了,他们那饿陷了的眼眶又葬埋了所有的希望。\\

		\item [$\bigtriangleup$] 河南一百十县(连沦陷县份在内),遭灾的就是这个数目,不过灾区有轻重而已,兹以河流来别:临黄河与伏牛山地带为最重,洪河汝河及洛河流域次之,唐河淮河流域又次之。\\

		\item [$\bigtriangleup$] 河南是地瘠民贫的省份,抗战以来三面临敌,人民加倍艰苦,偏在这抗战进入最艰难阶段,又遭天灾。今春(指旧历)三四月间,豫西遭雹灾,遭霜灾,豫南豫中有风灾,豫东有的地方遭蝗灾。入夏以来,全省三月不雨,秋交有雨,入秋又不雨,大旱成灾。豫西一带秋收之荞麦尚有希望,将收之际竟一场大霜,麦粒未能灌浆,全体冻死。八九月临河各县黄水溢堤,汪洋泛滥,大旱之后复遭水淹,灾情更重,河南就这样变成人间地狱了。\\

		\item [$\bigtriangleup$] 现在树叶吃光了,村口的杵臼,每天有人在那里捣花生皮与榆树皮(只有榆树皮能吃),然后蒸着吃。在叶县,一位小朋友对我说:“先生,这家伙刺嗓子!”\\

		\item [$\bigtriangleup$] 每天我们吃饭的时候,总有十几二十几个灾民在门口鹄候号叫求乞。那些菜绿的脸色,无神的眼睛,叫你不忍心去看,你也没有那些剩饭给他们。\\

		\item [$\bigtriangleup$] 今天小四饥死了,明天又听说友来吃野草中毒不起,后天又看见小宝冻死在寨外。可怜那些还活泼乱跳的下一代,如今都陆续的离开了人间。\\

		\item [$\bigtriangleup$] 最近我更发现灾民每人的脸都浮肿起来,鼻孔与眼角发黑。起初我以为是因饿而得的病症。后来才知是因为吃了一种名叫“霉花”的野草中毒而肿起来。这种草没有一点水分,磨出来是绿色,我曾尝试过,一股土腥味,据说猪吃了都要四肢麻痹,人怎能吃下去!灾民明知是毒物,他们还说:“先生,就这还没有呢!我们的牙脸手脚都是吃得麻痛!”现在叶县一带灾民真的没有“霉花”吃,他们正在吃一种干柴,一种无法用杵臼捣碎的干柴,所好的是吃了不肿脸不麻手脚。一位老夫说:“我做梦也没有想到吃柴火!真不如早死。”\\

		\item [$\bigtriangleup$] 牛早就快杀光了,猪尽是骨头,鸡的眼睛都饿得睁不开。\\

		\item [$\bigtriangleup$] 一斤麦子可以换二斤猪肉,三斤半牛肉。\\

		\item [$\bigtriangleup$] 在河南已恢复了原始的物物交换时代。卖子女无人要,自己的年轻老婆或十五六岁的女儿,都驮到驴上到豫东驮河、周家口、界首那些贩人的市场卖为娼妓。卖一口人,买不回四斗粮食。麦子一斗九百元,高粱一斗六百四十九元,玉米一斗七百元,小米十元一斤,蒸馍八元一斤,盐十五元一斤,香油也十五元。没有救灾办法,粮价不会跌落的,灾民根本也没有吃粮食的念头。老弱妇孺终日等死,年轻力壮者不得不铤而走险,这样下去,河南就不需要救灾了,而需要清乡防匪,维持地方的治安。\\

		\item [$\bigtriangleup$] 严冬到了,雪花飘落,灾民无柴无米无衣无食,冻馁交迫。那薄命的雪花正象征着他们的命运。救灾刻不容缓了。\\
	\end{description}
\end{quote}
