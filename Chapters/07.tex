\fancyhead[LO]{{\scriptsize 【温故一九四二】附录}} %奇數頁眉的左邊
\fancyhead[RO]{\thepage} %奇數頁眉的右邊
\fancyhead[LE]{\thepage} %偶數頁眉的左邊
\fancyhead[RE]{{\scriptsize 【温故一九四二】附录}} %偶數頁眉的右邊
\fancyfoot[LE,RO]{}
\fancyfoot[LO,CE]{}
\fancyfoot[CO,RE]{}
\chapter*{附录}
\addcontentsline{toc}{chapter}{\hspace{11mm}附录}
%\thispagestyle{empty}
温故一九四二、一九四三年时,除了这场大灾荒,还有这些年代所发生的一些杂事。这些杂事中,最感兴趣的,是从当时的《河南明国日报》上,看到两则离异声明。这证明大灾荒只是当年的主旋律,主旋律之下,仍有百花齐放的正常复杂的情感纠纷和日常生活。我们不能以偏概全,一叶知秋,瞎子摸象,让巴掌山挡住眼。这就不全面了。我们不能只看到大灾荒,看不到人的全貌。从这一点说,我们对委员长的指责,也有些偏激了。另外,我们从这两则离异声明中,也可以看到时代的进步。下边是全文:\\

\begin{quote}
	紧要启事\\

缘鄙人与冯氏结婚以来感情不和难以偕老经双方同意自即日起业已离异从此男婚女嫁\\

各听自便此启\\

张荫萍冯氏启\\

声明启示\\

鄙人旧历十二月初六日赴洛阳送货鄙妻刘化许昌人该晚逃走将衣服被褥零碎物件完全带走至今数日音信全无如此人在外发生意外不明之事与鄙人无干自此以后脱离夫妻关系恐亲友不明特此登报郑重声明偃师槐庙村中正西街门牌五号田光寅启\\

\begin{flushright}
	一九九三年十二月\\
	北京十里堡\\
\end{flushright}
\end{quote}

\begin{center}
	- 全书完 -
\end{center}