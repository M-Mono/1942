\fancyhead[LO]{{\scriptsize 【温故一九四二】第四章}} %奇數頁眉的左邊
\fancyhead[RO]{\thepage} %奇數頁眉的右邊
\fancyhead[LE]{\thepage} %偶數頁眉的左邊
\fancyhead[RE]{{\scriptsize 【温故一九四二】第四章}} %偶數頁眉的右邊
\fancyfoot[LE,RO]{}
\fancyfoot[LO,CE]{}
\fancyfoot[CO,RE]{}
\chapter*{四}
\addcontentsline{toc}{chapter}{\hspace{11mm}第四章}
%\thispagestyle{empty}
《大公报》被停刊三天。《大公报》停刊不怪《大公报》,全怪我故乡三千万灾民不争气。这些灾民中间,当然包括我姥娘一家,我二姥娘一家,我三姥娘一家,逃难的和留下的,饿死的和造反的,被狗吃的或被人吃的。虽然他们从来没有见过《大公报》。《大公报》重庆版于一九四三年二月一日刊载了他们在灾难中的各种遭遇。这激怒了委员长,于是下令停刊三天。当然,《大公报》这么做,一半是为了捕捉新闻,一半是出自中国知识分子的传统的被统治地位所带来的对劳苦大众的同情感。也许还有上层政治斗争牵涉到里面?这就不得而知了。他们派往灾区的记者叫张高峰。张高峰其人的个人历史、遭遇、悲欢,他的性格、为人及社会关系,虽然我很感兴趣,但根据我手头的资料,已无从考察,不过从文章中所反映出的个人品格,不失为一个素质优良、大概人到中年的男性。他在河南跑了许多地方,写了一篇前边曾引述过的《豫灾实录》,。这篇稿子共六千字左右。没想到这六千字的文章,竟在偌大一个中国引起麻烦。麻烦的根本原因,是因为这六千字里写了三千万人的真实情况。其实三千万人每个人的遭遇都可以写上几万字、几十万字,他只写了六千字,六千字除以三千万,每人才平均0.0002个字,这接近于0,等于没写。这竟引起了几亿人的委员长大发肝火。大发肝火的原因,许多人把其归罪于蒋的官僚主义。但如前所述,蒋绝不是不相信,而是他手头还有许多比这重大得多的国际国内政治问题。他不愿让三千万灾民这样一件小事去影响他的头脑。三千万灾民不会影响他的统治,而重大问题的任何一个细枝末节处理不当,他都可能地位不稳甚至下台;轻重缓急,他心中自有掂量,绝不是我们这些书生和草民所能理解的。三千万里死了三百万,十个里边才死了一个,死了还会生,生生死死,无法穷尽,何必操心?这是蒋委员长对《大公报》不满的根本点,也是这起新闻事件的症结。悲剧在于,双方仍存在误会。写文章的仍认为是委员长不了解实情,不实事求是;委员长一腔怒火,又不好明发出来,于是只好把复杂的事情简单处理:下令停刊。\\

《豫灾实录》里除了描述灾区人民的苦难,还同样如《时代》周刊记者白修德那样,写了逃出灾区的灾民的路上情况。两相对照,我们就可以相信这场灾难与灾民逃窜是真实的了。他写道,顺着陇海线逃往到陕西的灾民成千上万,扒上火车的男男女女像人山一样。沿途遗弃子女者日有所闻,失足毙命者是家常便饭。因为扒火车,父子姑嫂常被截为两伙,又遭到骨肉分离之苦。人人成了一副生理骨骼挂图。没扒火车步行逃难的,扶老携幼,独轮车父推子拉,六七十岁的老夫妻喘喘地负荷而行。“老爷,五天没吃东西啦!”他写道:\\

我紧闭起眼睛,静听着路旁吱吱的独轮车声,像压在我的身上一样。\\

他还写到狗吃人、人吃人的情形。\\

情形当然都是真实的。如果只是真实的情况,《大公报》也不会停刊。要命的是在二月一日刊载了这篇“实录”之后,二月二日,《大公报》主编王芸生,又根据这篇“实录”,结合政府对灾区的态度,写了一篇述评刊出,题目是《看重庆,念中原》,这才彻底打乱了蒋的思路,或者说,戳到了他的痛处,于是发火。\\

这篇述评说:\\

\begin{quote}
	\begin{description}
	\item [$\bigtriangleup$] 昨日本报登载一篇《豫灾实录》,想读者都已看到了。读了那篇通讯,任何硬汉都得下泪。河南灾情之重,人民遭遇之惨,大家差不多都已知道;但毕竟重到什么程度,惨到什么情形,大家就很模糊了。谁知道那三千万同胞,大都已深陷饥饿死亡的地狱。饿死的暴骨失肉,逃亡的扶老携幼,妻离子散,挤人丛,挨棍打,未必能够得到赈济委员会的登记证。吃杂草的毒发而死,啃干树皮的忍不住刺喉绞肠之苦。把妻女驮运到遥远的人肉市场,未必能够换到几斗粮食。这惨绝人寰的描写,实在令人不忍卒读。\\
	
	\item [$\bigtriangleup$] 尤其令人不忍的,灾荒如此,粮课依然。县衙门捉人逼捐,饿着肚皮纳粮,卖了田纳粮。忆旧时读杜甫所咏叹的《石壕吏》辄为之掩卷太息,乃不意竟依稀见到今日的事实。今天报载中央社鲁山电,谓“豫省三十一年度之征粮征购,虽在灾情严重下,进行亦颇顺利。”所谓:“据省田管处负责人谈,征购情形极为良好,各地人民均罄其所有,贡献国家”。这“罄其所有”四个字,实出诸血泪之笔。\\
	\end{description}
\end{quote}


文章接下去描写重庆物价跳涨,市场抢购,限价无限,而阔人豪奢的情况。然后说:\\

\begin{quote}
	\begin{description}
		\item [$\bigtriangleup$] 河南的灾民卖田卖人甚至饿死,还照纳国课,为什么政府就不可以征发豪商巨富的资产并限制一般富有者“满不在乎”的购买力?看重庆,念中原,实在令人感慨万千。\\
	\end{description}
\end{quote}

这篇社评发表的当天,委员长就看到了。当晚,新闻检查所派人送来了国民党政府军事委员会限令《大公报》停刊三天的命令。《大公报》于是二月三、四、五日停刊了三天。\\

对于王芸生其人,我也像对张高峰一样不甚了了。但从现有资料看,其人在当时与当局似过从甚密,与蒋的贴身人物陈布雷甚至蒋本人都有交往。但可以肯定,他毕竟只是一个办报的,并不理解委员长的处境和内心。不过对他写社评的这种稍含幼稚的勇气,就是放到今天,也不能不佩服。要命的是,《大公报》被停刊,王芸生感到很不理解,他认为,这篇文章不过尽写实任务之百一,为什么竟触怒委员长了呢?委员长提倡“民主”和“自由”,这不和他的口号相违背、公开压迫舆论了吗?为此,王芸生向陈布雷询问究竟,陈说了一段我们前边曾引述过的一段话。由于陈是蒋的贴身人物(侍卫室二组组长),这段话值得再引述一遍,由此可看出蒋的孤独和为难:\\

委员长根本不相信河南有灾,说是省政府虚报灾报。李主席(培基)的报灾电,说什么“赤地千里”,“哀鸿遍野”,“嗷嗷待哺”等等,委员长就骂是谎报滥调,并且严令河南的征实不得缓免。\\

可见连陈布雷也蒙在鼓里。陈的一番话,说得王芸生直眨巴眼。就像螺丝与螺母不但型号不同,连形状都不同所以根本无法对接一样,王芸生怪委员长不恤民命,其实责任不在蒋一方,而是王芸生不懂委员长的心。反过来,蒋心里对王肯定是极大的蔑视与看不起,怪他幼稚,不懂事,出门做事不令人放心。因此,在这篇社评发表之前,一九四二年末,美国国务院战时情报局曾约定邀请王芸生访美。经政府同意,发了护照,买了外汇,蒋介石宋美龄还为王芸生饯了行。飞机行期已定,这时王读到张高峰的报道,写了《看重庆,念中原》这篇文章。距出发的前两天,王芸生接到国民党中央宣传部长张道藩的电话,说:\\

“委员长叫我通知你,请你不要到美国去了。”\\

于是,王芸生的美国之行就作罢了。王、蒋之间,双方在不同层次、不同水平、不同想法之下,打了一场外人看来还很热闹、令人很义愤其实非常好笑和不得要领的交手仗。\\

可以肯定地说,《大公报》的灾区报道和社评,并没有改变蒋对灾区的已定的深思熟虑的看法和态度。采取的办法就是打板子、停报。知道这是从古到今对付文人的最好办法。文人的骨头是容易打断的。板子打了也就打了,报停了也就停了,美国之行不准也就不准了,接下去不会产生什么后果,唯一的效果是他们该老实了。所以,我与我故乡的三千万灾民,并不对张高峰的报道与王芸生的社评与呼喊表示任何感谢。因为他们这种呼喊并不起任何作用,惹怒委员长,甚至还起反面作用。我们可以为开他们,我们应该感谢的是洋人,是那个美国《时代》周刊记者白修德。他在一九四二、一九四三年的大灾荒中,真给我们这些穷人帮了忙。所谓帮忙,是因为这些帮忙起了作用,不起作用的帮忙只会给我们增加由希望再到失望的一个新的折磨过程。这也是委员长对待不同人所采取的不同态度。这说明蒋也不是一个过于固执的人,他也是可以变通的。对待国人,大家是他的治下,全国有几万万治下,得罪一个两个,枪毙一个两个,都不影响大局;书生总认为自己比灾民地位高,其实在一国之尊委员长心中,即使高,也高不到哪里去。但对待洋人就不同,洋人是一个顶一个的人,开罪一个洋人,就可能跟着开罪这个洋人的政府,所以得小心对待{\Add ──}这是在人与政府关系上,中国与外国的区别。白修德作为一个美国知识分子吧,看到“哀鸿遍野”,也激起了和中国知识分子相同的同情心与愤怒,也发了文章,不过不是发在中国,而是发到美国。文章发在美国,与发在中国就又有所不同。发在中国,委员长可以停刊;发在《时代》周刊,委员长如何让《时代》周刊停刊呢?白修德明确地说,如果不是美国新闻界行动起来,河南仍作为无政府状态继续存在。美国人帮了我们大忙。当我们后来高呼“打倒美帝国主义”时,我想不应该忘记历史,起码一九四二年、一九四三年这两年不要打倒。白修德在灾区跑了一圈后,就迫不及待地想把灾区的消息发出去。所以在归途中的第一个电报局{\Add ──}洛阳电报局{\Add ──}就草草地发了电稿。按照当时重庆政府的规定,新闻报道是要通过中宣部检查的。如果一经检查,这篇报道肯定会被扣压;然而,这封电报却从洛阳通过成都的商业电台迅速发往了纽约。或者是因为这个电台的制度不严(对于一个专制国家来说,制度不严也不失为一个好事),或者是因为洛阳电报局某一位报务员良心发现,这篇报道不经检查就到达了纽约。于是,消息就通过《时代》杂志传开了。宋美龄女士当时正在美进行那次出名的访问。当她看到这篇英文报道后,十分恼火;也是一时心急疏忽,竟在美国用起了中国的办法,要求《时代》周刊的发行人亨利·卢斯把白修德解职。当然,她的这种中国式的要求,理所当然地被亨利·卢斯拒绝了。那里毕竟是个新闻自由的国度啊。别说宋美龄,就是揭了罗斯福的丑闻,罗斯福夫人要求解雇记者的做法,也不一定会被《时代》周刊当回事。须知,罗当总统才几年?《时代》周刊发行多少年了?当然,我想罗夫人也不会这么蠢,也不会产生这么动不动就用行政干涉的思路和念头。\\

一夜之间,白修德在重庆成了一个引起争论的人物。一些官员指责他逃避新闻检查;另一些官员指控他与电报局里的共产党员密谋。但不管怎样,他们都对白修德奈何不得,这是问题的关键。这时,白修德已通过美国陆军情报机构把情况报告了史迪威。也报告了美国驻华大使馆。还报告了中国的国防部长。还见到了中国的立法院院长,四川省主席,孙中山博士的遗孀宋庆龄{\Add ──}白修德这样广泛地动员社会力量,是任何一个中国记者或报纸主编都难以办到的。\\

中国国防部长的态度是:\\

“白修德先生,如果不是你在说谎,就是别人在对你说谎!”\\

立法院长、四川省主席都告诫白修德,找他们这些人是白找,只有蒋介石说话,才能起作用,中国大地上才能看到行动。\\

但见蒋是不容易的。通过宋庆龄的帮助,花了五天时间,白修德才见到蒋。如果没有孙的夫人、蒋的亲属帮忙,一切就要拉吹(所以,在专制制度下,裙带关系也不一定全是不正之风,有时也是为民请命之风)。据白修德印象,孙夫人风姿优雅、秀丽。她说:\\

“据悉,他(蒋介石)在长时间单调的外出视察后非常疲倦,需要休息几天。但我坚持说,此事关系到几百万的生命问题……我建议你向他报告情况时要像你向我报告时那样坦率无畏。如果说一定要有人人头落地的话,也不要畏缩。……否则,情况就不会有所改变。”\\

蒋介石在他那间阴暗的办公室接见了白修德,见面时直挺着瘦长的身子,面色严峻,呆板地与白修德握了握手,然后坐在高靠背的椅子上,听白修德谈话。白修德记载,蒋在听白修德申诉时,带着明显的厌恶神情。白修德把这理解成蒋的不愿相信,这说明白修德与中国文人犯了同样的错误。他们没有站在同一层次上对话。他们把蒋理解得肤浅得多。蒋怎么会不相信呢?蒋肯定比白更早更详细地知道河南灾区的情况,无非,这并不是他手头的重要事情。现在一些低等官员、中国文人、外国记者,硬要把他们认为重要其实并不重要的事情当做重要的事情强加在他头上,或者说把局部重要的事情当成全局重要的事情强加在他头上,不答应就不罢休,还把文章从国内登到国外,造成了世界舆论,把不重要的局部的事情真闹成了重要的全局的事情,使得他把对他来讲更重要的事情放到一边,来听一个爱管闲事的外国人向他讲述中国的情况,真是荒唐,让人又好气又好笑;好比一个大鹏,看蓬间雀在那里折腾,而且真把自己折腾进去,扯到一堆垛草和乱麻之中时的心情。他不知为什么这么多双不同形状、不同肤色的手,都要插到这狗屎堆里。这才是他脸上所露出的厌恶表情的真正含义。这含义是白修德所不理解的,一直误会了五十年。人与人之间,是多么难以沟通啊。蒋听得无聊,只好没话找话,对他的一个助手说:\\

“他们(指灾区老百姓)看到外国人,什么话都会讲。”\\

白修德接下去写道:\\

显然,他并不知道正在发生的这些事情。\\

这就是白修德的自作聪明和误会之处了。不过中国的事情也很有意思。如果不误会,白修德就没有这么大的义愤,没有这么大的义愤,就不会直逼蒋介石;而这种误会和直逼,还真把这么大智能大聪明整天考虑大事的蒋给逼到了墙角。因为问题在于:蒋一切明白,但他身有大事;可他作为一国之君,又不能把三千万这个小事当做小事说出来;如果说出来,他成了什么形象?这是蒋的难言之隐。而白修德的直逼,正逼在蒋的难言之隐上,所以蒋也是哭笑不得,而白也真把蒋当做不了解情况。白找到这样一个谈话的突破口,即说河南灾区在发生人吃人的情况。蒋听到这个消息,也以为白修德这样的美国人不会亲自吃苦到灾区跑那么多地方,见那么多事情,估计也是走马观花,胡乱听了几耳朵,于是赶忙否认,说:\\

“白修德先生,人吃人的事在中国是不可能的!”\\

白修德说:\\

“我亲眼看到狗吃人!”\\

蒋又赶忙否认:\\

“这是不可能的!”\\

这时白修德便将等候在接待室的英国《泰晤士》报记者福尔曼叫了进来,将他们在河南灾区拍的照片,摊到了委员长面前。几张照片清楚地表明,一些野狗正站在沙土堆里扒出来的尸体上。这下将蒋委员长震住了。白修德写到,“他看到委员长的两膝轻微地哆嗦起来,那是一种神经性的痉挛”。我想,这时的委员长首先是恼怒,对白修德及福尔曼的恼怒,对灾区的恼怒,对各级官员的恼怒,对这不重要事情的恼怒,对世界上重要事情的恼怒,正是那些重要事情的存在,才把这些本来也重要的事情,逼得不重要了;如果不是另外有更重要的事情存在,他也可以动员全国人民一起抗灾,到灾区视察、慰问,落下一个爱民如子的好印象。但他又不能把这一切恼怒发泄出来,特别不能当着外国记者发泄出来。于是只好对着真被外国人搞到的狗吃人的照片痉挛、哆嗦,像所有的中国统治者一样,一到这时候,出于战略考虑,态度马上来了一个一百八十度的大转弯,做出严肃的样子,做出以前不了解情况现在终于了解情况还对提供情况人有些感激终于使他了解真相的样子,马上拿出小纸簿和毛笔,开始做记录,让白修德和福尔曼提供一些治灾不力的官员的名字{\Add ──}这也是中国统治者对付事情的惯例,首先从组织措施上动刀子,接着还要求提供另一些人的名字:要他们再写一份完整的报告。然后,正式向他们表示感谢,说,他们是比政府“派出去的任何调查员”都要好的调查员。接着,二十分钟的会见就结束了,白修德和福尔曼被客客气气地送出去了。\\

我想,白二人走后,蒋一定摔了一只杯子,骂了一句现在电影上常见的话:\\

“娘希匹!”\\

很快,由于一张狗吃人的照片,人头开始像宋庆龄预料地那样落地了。不过是以给白修德提供方便向美国传稿的洛阳电报局那些不幸的人开始的。因为他们让河南饿死人那样令人难堪的消息泄露到了美国。但是,也有许多生命得救了。白修德写道:是美国报界的力量救了他们。白写这句话时,一定洋洋自得;我引述这句话时,心里却感到好笑。不过,别管什么力量,到底把委员长说服了,委员长动作了;委员长一动作,许多生命就得救了。谁是我们的救星呢?谁是灾民的救星呢?说到底,还是一国之尊的委员长啊。虽然这种动作是阴差阳错、万般误会导致的。但白修德由于不通中国国情,仍把一切功劳揽到自己身上。他不明白,即使美国报界厉害,但那只是诱因,不是结果;对于中国,美国报界毕竟抵不过委员长啊。但白洋洋自得,包括那些在华的外国主教。白修德这时在重庆收到美国主教托马斯·梅甘从洛阳发来的一封信:\\

你回去发了电报以后,突然从陕西运来了几列车粮食。在洛阳,他们简直来不及很快地把粮食卸下来。这是头等的成绩,至少说是棒球本垒打出的那种头等成绩。省政府忙了起来,在乡间各处设立了粥站。他们真的在工作,并且做了一些事情。军队从大量的余粮中拿出一部分,倒也帮了不少忙。全国的确在忙着为灾民募捐,现款源源不断地送往河南。\\

在我看来,上述四点是很大的成功,并且证实了我以前的看法,即灾荒完全是人为的,如果当局愿意的话,他们随时都有能力对灾荒进行控制。你的访问和对他们的责备,达到了预期的目的,使他们惊醒过来,开始履行职责,后来也确实做了一些事情。总之,祝愿《时代》和《生活》杂志发挥更大的影响,祝愿《幸福》杂志长寿、和平!这是了不起的!……在河南,老百姓将永远把你铭记在心。有些人心憎爱分明十分舒畅地怀念你,但也有一些人咬牙切齿,他们这样做是不奇怪的。\\