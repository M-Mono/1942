\fancyhead[LO]{{\scriptsize 【温故一九四二】第二章}} %奇數頁眉的左邊
\fancyhead[RO]{\thepage} %奇數頁眉的右邊
\fancyhead[LE]{\thepage} %偶數頁眉的左邊
\fancyhead[RE]{{\scriptsize 【温故一九四二】第二章}} %偶數頁眉的右邊
\fancyfoot[LE,RO]{}
\fancyfoot[LO,CE]{}
\fancyfoot[CO,RE]{}
\chapter*{二}
\addcontentsline{toc}{chapter}{\hspace{11mm}第二章}
%\thispagestyle{empty}
重庆黄山官邸。这里生机盎然,空气清新,一到春天就是满山的桃红和火焰般的山茶花。自南京陷落以后,国民政府迁都重庆,这里是蒋介石委员长的住处。当时蒋在重庆有四处官邸,这是其中之一。领袖的官邸,与国家沦陷、国家强弱没有关系;这里既不比南京的几处官邸差,也不比美国的白宫、英国的唐宁街十号逊色。领袖总是领袖,只要能当上领袖,不管当上什么肤色、民族的领袖,都可以享受到世界一流的衣、食、住、行。虽然所统治的民众大相径庭。所以,我历来赞成各国领袖之间握手言欢,因为他们才是真正的阶级兄弟;各国民众之间,既不必联合,也没有什么可说的。即使发生战争,也不可怕,世界上最后一颗炮弹,才落在领袖的头上。如果发生世界性的核战争,最后剩下的,就是各国的几位领袖,因为他们这时住在风景幽美的地球上空,掌握着核按钮。掌握按纽的人,历来是不会受伤害的。黄山官邸以云岫楼和松厅为中心结构,蒋住云岫楼,仪态万方的宋美龄住松厅。当然,夜间就难说了,如果两人有兴致的话。在两处住宅之间的低谷里,专门挖有防空洞,供蒋、宋躲他们阶级兄弟日本天皇陛下的飞机。至于蒋、宋的日常生活,这不是我们所能想象的,反正整日的吃喝,比五十年后我们十二亿人中的十一亿九千九百九十九万人还要好,还要不可想象。虽然蒋只喝白水,不饮酒、不抽烟、安假牙,信基督,但他也肯定知道,榆树皮和“霉花”,是不可吃的,可吃的是西餐和中餐中的各种菜系。一九四二年,蒋与他的参谋长、美国人史迪威发生矛盾,在黄山官邸吵嘴,即要不欢而散,宋美龄挽狂澜于即倒,美丽地笑着说:“将军,都是老朋友了,犯不着这样怄气。要是将军能赏光到我的松厅别墅去坐一坐,将会喝到可口的咖啡!”\\

这是我在一本书上读到的。读到这里,我对他们吵不吵嘴并不感兴趣,反正吵嘴的双方都已经去球了,不在这个世界上了。我注意到:一九四二年,中国还是有“可口的咖啡”,虽然我故乡的人民在吃树皮、柴火、稻草和使人身体中毒发肿的“霉花”,最后饿死三百万人。当然,这样来故意对比,说明我这个人无聊,把什么事情都弄得庸俗化。我也知道,对一个泱泱大国政府首脑的要求,不在他的夫人有无有咖啡,只要他们每天不喝人血(据说中非的皇帝就每天喝人血),无论喝什么,吃什么,只要能把国家治理好,就是一个民族英雄和历史伟人。我在另一本书上看到,蒋为了拉拢一部地方武装,对戴笠说:“你去办一办。记住,多花几个钱没关系。”这钱从何而来呢?我只是想说,一九四二年,当我故乡发生大旱灾、大饥饿的消息传到黄山官邸时,蒋委员长对这消息不该不相信。当然,也不是不信,也不是全信,他说:可能有旱灾,但情况不会这么严重。他甚至怀疑是地方官员虚报灾情,像军队虚报兵员为了吃空额一样,想多得一些救济粮和救济款。蒋委员长的这种态度,在几十年后的今天,受到许多书籍的指责。他们认为委员长不体察民情、不爱民如子、固执等。他们这种爱民如子、横眉冷对民贼独夫的态度,也感染了我的情绪。但当我冷静下来,我又是轻轻一笑。这时我突然明白,该受指责的不是委员长,而是几十年后这些书的自作聪明的作者。是侍从在梦中,还是丞相在梦中?侍从在梦中。不设身处地,不身居高位,怎么能理解委员长的心思?书籍的作者,不都是些百无一用的书生吗?委员长是委员长都当上了,头脑不比一个书生聪明?是书生领导委员长,还是委员长领导书生?是委员长见多识广,还是书生见多识广?一切全在委员长{\Add ──}万般世界,五万万百姓,皆在委员长心中。只是,当时的委员长的所思所想,高邈深远,错综复杂,并不被我们所理解。委员长真不相信河南有大旱灾、旱灾会饿死人吗?非也。因为从委员长的出身考察,相对于宋美龄小姐来说,委员长还算是苦出身。委员长自己写道:\\

我九岁丧父……。当时家里的悲惨情况实在难以形容。我家无依无靠,没有势力,很快成了大家污辱和虐待的对象。\\

这样一个出身的人,不会不知道下层大众所遭受的苦难。在一个省的全部范围内发生了大旱灾,情况严重到什么程度,他心里不会没底。但他认为:可能有旱灾,但不会这么严重。于是书生们上了当,以为委员长是官僚主义。其实在梦中的是书生,清醒的是委员长。那么为什么心里清楚说不清楚呢?明白情况严重而故意说不严重呢?这是因为摆在他面前的,有更多的,比这个旱灾还严重的混沌不清需要他理清楚处理妥当以致不犯历史错误的重大问题。须知,在东方饿死三百万人不会影响历史。\\

这时的委员长,已不是一个乡巴佬,而是一个领袖。站在领袖的位置上,他知道轻重缓急。当时能导致历史向不同方向发展的事情大致有:\\

\begin{quote}
	\begin{description}
		\item [一、] 	中国的同盟国地位问题。当时同盟国有美、英、法、苏、中等。蒋虽是中国的领袖,但同盟国的领袖们坐在一起开会,如开罗会议,蒋就成了一个普通人,成了一个小弟兄,成了一个无足轻重的人。大家在一起,似乎罗斯福、丘吉尔、斯大林,都不把蒋放在眼里。不把蒋放眼里,就是不把中国放到眼里。由此以来,在世界战局的分布上,中国就常常是战略的受害者。而中国最穷,必须在有外援的情况下才能打这场战争,所以常常受制于人,吃哑巴亏;带给蒋个人的,就是仍受“侮辱和虐待”。这是他个人心理上暗自痛恨的。\\
		
		\item [二、] 	对日战争问题。在中国正面战场,蒋的军队吸引了大部分在华日军,虽然不断丢失土地,但从国际战略上讲,这种牵制本身,就给其它同盟国带来莫大的利益;但同盟国其它领袖并没认清这一点或是认清了这一点而故意欺辱人,所给的战争物资,与国民党部队所担负的牵制任务,距离相差非常大;从国内讲,国民党部队在正面战场牵制日军,使得共产党在他的根据地得到休养生息,这是蒋的心腹大患,于是牵涉到了对共产党的方针。蒋有一著名的理论,“攘外必先安内”。这口号从民族利益上讲,是狭隘的,容易激起民愤的;如果从蒋的统治利益出来,又未尝不是一个统治者必须采取的态度。如只是攘外,后方的敌人发展起来,不是比前方的敌人更能直捣心脏吗?关于这一方针,他承受着巨大的国际、国内压力。\\
		
		\item [三、] 	国民党内部、国民政府内部各派系的斗争。蒋曾很后悔地说:北伐战争之后,他不该接受那么多军阀部队;一九四九年后说:我不是被共产党打倒的,我是被国民党打倒的;可见平日心情。\\
		
		\item [四、] 	他与他的参谋长{\Add ──}美军上将史迪威将军,发生了严重的战略上和个人间的矛盾,这牵涉到对华援助和蒋个人在美国的威信问题。史迪威已开始在背后不体面地称这位中国民族的领袖为“花生米”{\Add ──}以上所有这些问题,包括一些我们还没觉察到而蒋在他的位置上已经觉察到的问题,都有可能改变历史的方向和写法,这时,出现了一个地方省(当时全国三十多个省)的旱灾,显得多么无足轻重。死掉一些本就无用、是社会负担的老百姓,不会改变历史的方向;而他在上层政治的重大问题上处理稍有不慎,历史就可能向不利于他的方向发展,后来一九四五年至一九四九年,就证明了这一点。上述哪一个重大问题,对于一个领袖来讲,都比三百万人对他及他的统治地位影响更直接,更利益交关。从历史地位上说,三百万人确没有一粒“花生米”重要。所以,他心里清楚旱灾,仍然要说:可能有旱灾,但不会那么严重。于是他厌恶那些把他当傻瓜当官僚以为他不明真相而不厌其烦向他提供真情况的人,特别是那些爱管闲事、爱干涉他国内政的外国人。这就是蒋委员长此时此刻的心境。当然,这是站在蒋的立场上考察问题;如果换一个角度,当我们站在几千万灾民的立场上去考察,就觉得蒋无疑是独夫民贼,置人民的生死于不顾了。\\
	\end{description}
\end{quote}

世界有这样一条真理,一旦与领袖相处,我们这些普通的百姓就非倒霉不可。蒋的这种态度,使受灾的几千万人只有吃树皮、稻草、干柴和“霉花”,而得不到一个政府所应承担的救济,调剂和帮助。于是,人口在大面积死亡。但这不是事情最重要的部分,事情最重要的部分是:\\

在大面积受灾和饿死人的情况下,政府向这个地区所征的实物税和军粮任务不变。\\

陈布雷说:\\

委员长根本不相信河南有灾,说是省政府虚报灾情。李主席(培基,河南省政府主席)的报灾电,说什么“赤地千里”,“哀鸿遍地”“嗷嗷待哺”等等,委员长就骂是谎报滥调,并且严令河南的征实不能缓免。\\

这实际等于政府又拿了一把刀子,与灾害为伍,在直接宰杀那些牲口一样的两眼灰蒙蒙、东倒西歪的灾民。于是,死的死了;没死的,发生大面积背井离乡的逃荒。五十年后的今天,我们也会像蒋委员长那样说:情况不会那么严重吧?这是一种事物的惯性,事物后特别过很长一段时间后再来想事物,我们总是宽宏大量地想:事情不会那么严重吧?但在当时,可知历史是一点不宽容的。为了证明这一点,我们又得引用资料。我认为这种在历史中打捞事件的报告式的文字,引用资料比作者胡编乱造要更科学一些。后者虽然能使读者身临其境,但其境是虚假的;资料也可能虚假,但五十年前的资料,总比五十年后的想象更真实一些。一九四二年,美国驻华外交官约翰·S·谢伟思在给美国政府的报告中写道:\\

河南灾民最大的负担是不断加重的实物税和征收军粮。由于在中条山失陷之前,该省还要向驻守山西南部的军队和驻守在比较穷困的陕西省的军队提供给养,因而,负担也就更加沉重了。在陕西省的四十万驻军的主要任务是“警戒”共产党。\\

我从很多人士那里得到的估计是:全部所征粮税占农民总收获的百分之三十至五十,其中包括地方政府的征税,全国性的实物土地税(通过省政府征收)以及形形色色、无法估计的军事方面的需求。税率是按正常的年景定,而不是按当年的实际收成定。因此,收成越坏,从农民征收的比例就越大。征粮要缴纳小麦,因此,他们所收获的小麦更大一部分要用于纳粮。\\

有很可靠的证据表明,向农民征收的军粮是超过实际需要的。中国军官的一个由来已久的,仍然盛行不衰的惯例,就是向上级报告的部队人数超过实际所有的人数。这样他们就可以吃空额,谋私利。洛阳公开市场上的很大一批粮食,就是来自这个方面……\\

人们还普遍抱怨,征粮征税负担分配不公平。这些事是通过保甲长来办,他们自己就是乡绅、地主。他们通常都是要使自己和他们的亲朋好友不要纳粮纳税太多。势力还是以财富和财产为基础:穷苦农民的粮食,往往被更多地征去了,这就正像是他们的儿子,而不是甲长和地主的儿子,被拉去当兵一样。\\

河南的情况是如此之糟,以致在好几年中都有人逃荒到陕西、甘肃和川北……。结果是河南的人口相对减少,而留下来的,人和赋税负担相对加重了。在前线地区,农民的日子最苦,那里受灾也最重。因此,来自那里的人口流动也最多。来自郑州的一位传教士说,早在当年的饥荒袭来之前,那个地区的许多田园就已荒无人烟了。\\

这种情况今年发展到了顶点。最盲目的政府官员也认识到,在小麦欠收后,早春将发生严重缺粮。早在七月间,每天就有约一千名难民逃离河南,但是,征粮计划不变。在很多地区,全部收成不够纳粮的需要。在农村发生了一些抗议,但都是无力的,分散的,没有效果的。在少数地方,显然使用了军队对付人民。吃着榆树皮和干树叶的灾民,被迫把他们最后一点粮食种子交给税收机关。身体虚弱得几乎走不动路的农民还必须给军队交纳军马饲料。这些饲料比起他塞进自己嘴里的东西,其营养价值要高得多。\\

以上是谢伟思的报告。为什么我引用谢的文字而不引证别的书籍呢?因为谢是外国人,不身在复杂的其中,也许能更客观一些。但谢伟思所说的,还不是最严重的,即:在灾难中的灾民,并不被免除赋税,而是严令仍按正常年景税赋征收因而实际上税赋已超过正常年景还不是重要的,更重要的,是统治这些灾民的一些官员,还借灾民的灾难去投机发财。据美国记者白修德亲眼目睹,有些部队的司令把部队的余粮卖给实民,发了大财。来自西安和郑州的商人,政府的小官吏、军官以及仍然储蓄着粮食在手的地主,拼命以罪恶的低价收买农民祖辈留下来的田地。土地的集中和丧失同时进行,其激烈程度与饥饿的程度成正比。\\

当我们被这么一些从委员长一直到小官吏、地主所统治的时候,我们的命运操纵在他们手里,我们对他们的操纵能十分放心吗?\\

后来,就必然出现了大批的脱离了土地的灾民,出现一个由东向西的大规模的流民图。这流民中,就包括河南延建县王楼乡老庄的俺二姥娘、俺三姥娘全家,包括村里其它许多父老乡亲。他们虽然一辈子没有见过委员长,许多青壮年一听委员长还自觉立正,但是,委员长在富丽堂皇的黄山别墅的态度,一颦一笑,都将直接决定他们的生死和命运。委员长思索:中国向何处去?世界向何处去?他们思索,我们向哪里去逃荒?\\