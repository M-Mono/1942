\fancyhead[LO]{{\scriptsize 【温故一九四二】序}} %奇數頁眉的左邊
\fancyhead[RO]{\thepage} %奇數頁眉的右邊
\fancyhead[LE]{\thepage} %偶數頁眉的左邊
\fancyhead[RE]{{\scriptsize 【温故一九四二】序}} %偶數頁眉的右邊
\fancyfoot[LE,RO]{}
\fancyfoot[LO,CE]{}
\fancyfoot[CO,RE]{}
\chapter*{序}
\addcontentsline{toc}{chapter}{\hspace{11mm}序}
%\thispagestyle{empty}
一九四二年,河南发生大灾荒。一位我所敬重的朋友,用一盘黄豆芽和两只猪蹄,把我打发回了一九四二年。当然,这顿壮行的饭,如果放到一九四二年,可能是一顿美味佳肴;同时就是放到一九四二年,也不见得多么可观。一九四三年二月,美国《时代》周刊记者白修德、英国《泰晤士》报记者哈里逊·福尔曼去河南考察灾情,在母亲煮食自己婴儿的地方,我故乡的省政府官员,宴请两个外国友人的菜单是:莲子羹、胡椒辣子鸡、栗子炖牛肉、豆腐、鱼、炸春卷、热馒头、米饭、两道汤,外加三个撒满了白糖的馅饼。这饭就是放到今天,我们这些庸俗的市民,也只能在书中和大饭店的菜本上看到。白修德说:这是他所吃过的最好的筵席之一。我说:这是我看到的最好的筵席之一。但他又说:他不忍心吃下去。我相信我故乡的省政府官员,决不会像白修德这么扭扭捏捏。说到底,一九四二年至一九四三年,我故乡发生了吃的问题。但吃的问题应该仅限在我们这些普通的百姓身上。我估计在我们这个东方文明的古国,无论发生什么情况,县以上的官员,都不会发生这种问题。不但不存在吃的问题,性的问题也不会匮乏。\\

还有一个问题,当我顺着枯燥泛出霉尿味的隧道回到一九四二年时,我发现五十年后我朋友把他交给我的任务的重要性,人为地夸大了。吃完豆芽和猪蹄,他是用一种上校的口气,来说明一九四二年的。\\

一九四二年夏到一九四三年春,河南发生大旱灾,景象令人触目惊心。全省夏秋两季大部绝收。大旱之后,又遇蝗灾。灾民五百万,占全省人口的百分之二十。“水旱蝗汤”,袭击全省一百一十个县。\\

灾民吃草根树皮,饿殍遍野。妇女售价累跌至过去的十分之一,壮丁售价也跌了三分之一。寥寥中原,赤地千里,河南饿死三百万人之多。\\

死了三百万。他严肃地看着我。我心里也有些发毛。但当我回到一九四二年时,我不禁哑然失笑。三百万人是不错,但放在当时的历史环境中去考察,无非是小事一桩。在死三百万的同时,历史上还发生着这样一些事:宋美玲访美、甘地绝食、斯大林格勒大血战、丘吉尔感冒。这些事件中的任何一桩,放到一九四二年的世界环境中,都比三百万要重要。五十年之后,我们知道当年有丘吉尔、甘地、仪态万方的宋美龄、斯大林格勒大血战,有谁知道我的故乡还因为旱灾死过三百万人呢?当时中国国内形势,国民党、共产党、日军、美国人、英国人、东南亚战场、国内正面战场、陕甘宁边区,政治环境错综复杂,如一盆杂拌粥相互搅和,摆在国家最高元首蒋介石委员长的桌前。别说是委员长,换任何一个人,处在那样的位置,三百万人肯定不是他首先考虑的问题。三百万是三百万人自己的事。所以,朋友交给我的任务是小节而不是大局,是芝麻而不是西瓜。当时世界最重要的部分是白宫、唐宁街十号、克里姆林宫、希特勒的地下掩体指挥部、日本东京,中国最重要的部分是重庆黄山官邸。这些富丽堂皇地方中的衣着干净、可以喝咖啡洗热水澡的少数人,将注定要决定世界上大多数人的命运。但这些世界的轴心我将远离,我要蓬头垢面地回到赤野千里、遍地饿殍的河南灾区。这不能说明别的,只能说明我从一九四二年起,就注定是这些慌乱下贱的灾民的后裔。最后一个问题是:朋友在为我壮行时,花钱买了两只猪蹄。匆忙之中,他竟忘记拔下盘中猪蹄的蹄甲;我吃了带蹄甲的猪蹄,就匆匆上路;可见双方是多么大意。\\