\fancyhead[LO]{{\scriptsize 【温故一九四二】第三章}} %奇數頁眉的左邊
\fancyhead[RO]{\thepage} %奇數頁眉的右邊
\fancyhead[LE]{\thepage} %偶數頁眉的左邊
\fancyhead[RE]{{\scriptsize 【温故一九四二】第三章}} %偶數頁眉的右邊
\fancyfoot[LE,RO]{}
\fancyfoot[LO,CE]{}
\fancyfoot[CO,RE]{}
\chapter*{三}
\addcontentsline{toc}{chapter}{\hspace{11mm}第三章}
%\thispagestyle{empty}
重庆黄山官邸。这里生机盎然,空气清新,一到春天就是满山的桃红和火焰般的山茶花。自花爪舅舅直到现在还有些后悔。当初在洛阳被抓了壮丁,后来为什么要逃跑,没有在部队坚持下来呢?我问:\\

“当时抓你的是哪个部队?”\\

花爪舅舅:\\

“国军。”\\

我:\\

“我知道是国军,国军的哪一部分?”\\

花爪舅舅:\\

“班长叫个李狗剩,排长叫个闫之栋。”\\

我:\\

“再往上呢?”\\

花爪舅舅:\\

“再往上就不知道了。”\\

我事后查了查资料,当时占据洛阳一带的国民党部队,隶属胡宗南。我问:\\

“被抓壮丁后干什么去了?”\\

花爪舅舅:\\

“当时就上了中条山,派到了前线。日本人的追击炮,‘啾啾’地在头上飞。打仗头一天,班副和两个弟兄就被炸死了。我害怕了,当晚就开溜了。现在想起来,真是后悔。”\\

我:\\

“是呀,大敌当前,民族矛盾,别的弟兄牺牲了,你开溜了,是不大象话,该后悔。”\\

花爪舅舅瞪我一眼:\\

“我不是后悔这个。”\\

我一愣:\\

“那你后悔什么?”\\

花爪舅舅:\\

“当初不开溜,后来跑到台湾,现在也成台胞了。像通村的王明芹,小名强驴,抓壮丁比我还晚两年,后来到了台湾,现在成了台胞,去年回来了,带着小老婆,戴着金壳手表,镶着大金牙,县长都用小轿车接他,是玩的不是?这不能怪别的,只能怪你舅眼圈子太小,年轻不懂事。当时我才十五六岁,只知道活命了。”\\

我明白了花爪舅舅的意思。我安慰他:\\

“现在后悔是对的,当初逃跑也是对的。你想,一九四三年,离抗日战争结束还有两年,以后解放战争还有五年,谁也难保证你在诸多的战斗中不像你们班副一样被打死。当然,如果不打死,就像强驴一样成了台胞;如果万一打死,不连现在也没有了。”\\

花爪舅舅想了想:\\

“那倒是,子弹没长眼睛;我就是这个命,咱没当台胞那个命。”\\

我说:\\

“你虽然没当台胞,但在咱们这边,你也当了支书,总起说混得还算不错。”\\

花爪舅舅立即来了精神:\\

“那倒是,支书我一口气当了二十四年!”\\

但马上又颓然叹口气:\\

“但是十个支书,加起来也不顶一个台胞呀。现在又下了台,县长认咱是谁呀。”\\

我安慰他:\\

“认识县长也没什么了不起,不就是一个强驴吗?舅舅,咱们不说强驴了,咱们说说,俺二姥娘一家、三姥娘一家,当初是怎么逃荒的,你身在其中,肯定有许多亲身经历。”\\

一说到正题,花爪舅舅的态度倒变得无所谓,叙述得也简单和枯燥了。两手相互抓着说:\\

“逃荒就是逃荒呗。”\\

我:\\

“怎么逃荒,荒怎么逃法?”\\

他:\\

“俺爹推着独轮车,俺二大爷挑着箩筐,独轮车上装些锅碗瓢盆,箩筐里挑些小孩。路上拉棍要饭,吃树皮,吃杂草。后来到了洛阳,我就被抓了兵。”\\

我不禁埋怨:\\

“你也说得太简单了,路上就没有什么现在还记得的事情?”\\

他眨眨眼:\\

“记得路边躺着睡觉特冷,半夜就冻醒了。见俺爹俺娘还在睡,也不敢说话。”\\

我:\\

“后来怎么抓的兵?”\\

他:\\

“洛阳有天主教办的粥场,我去挤着打粥,回来路上,就被抓了兵。”\\

我:\\

“抓兵俺三姥爷三姥娘知道不?”\\

他摇摇头:\\

“他们哪里知道?认为我被人拐跑了。再见面就是十年之后了。”\\

我点点头。又问:\\

“你抓兵他们怎么办?”\\

他:\\

“十年后我才听俺娘说,他们扒火车去陕西。扒火车时,俺爹差点让火车轧着。”\\

我:\\

“俺二姥娘家一股呢?”\\

他:\\

“你二姥爷家扒火车时,扒着扒着,火车就开了,把个没扒上来的小妹妹──你该叫小姨,也给弄失散了,直到现在没找见。”\\

我点点头。又问:\\

“路上死人多吗?”\\

他:\\

“怎么不多,到处是坟包,到处是死人。扒火车还轧死许多。”\\

我:\\

“咱家没有饿死的?”\\

他:\\

“怎么没有饿死的,你二姥爷,你三妗,不都是饿死在道儿上?”\\

我:\\

“就没有一些细节?”\\

这时花爪舅舅有些不耐烦了,愤怒地瞪我一眼:\\

“人家人都饿死了,你还要细节!”\\

说完,丢下我,独自蹶蹶地走了,把我扔在一片尴尬之中。这时我才觉得朋友把我打发回一九四二年真是居心不良,我在揭亲人和父老的已经愈合五十年的伤疤,让他们重新露出血淋淋的创面;何况这疤疖也结得太厚,被岁月和灰尘风干成了盔甲,搬动它像搬动大山一样艰难费劲。\\

没有风,太阳直射在一大溜麦秸垛上。麦秸垛旁显得很温暖。我蹲在麦秸垛旁,正费力地与一个既聋又瞎话语已经说不清楚且鼻涕流水的八十多岁的老人说话。老人叫郭有运。据县政协委员韩给我介绍,他是一九四三年大逃荒中家中受损失最重的一个。老婆、老娘、三个孩子,全丢在了路上。五年后他从陕西回来,已是孤身一人。现在的家庭,属于重起炉灶。但看麦秸垛后他重搭的又经营四十多年的新炉灶,证明他作为人的能力,还属上乘。因为那是我故乡乡村中目前还不常见的一幢不中不西的二层小楼。但如果从他年龄过大而房子很新的角度来考察,这不应算是他的能力,成绩应归功于坐在我们中间当翻译的留着分头戴着“戈尔巴乔夫”头像手表的四十岁的儿子。他的儿子一开始对我的到来并不欢迎,只是听说我与这个乡派出所的副所长是光屁股同学,才对我另眼相看。但听到我的到来与现实中的他没有任何关联,而是为了让他爹和我共同回到五十年前,而五十年前他还在风里云里飘,就又有些不耐烦。老人家的嘴漏风,呜里呜啦,翻译不耐烦,所得的五十年前的情况既生硬又零碎。我又一次深深体会到,在活人中打捞历史,实在不是一件容易的事。郭有运在一九四三年逃荒中的大致情况是:一上路,他娘就病了;为了给他娘冶病,卖掉一个小女;为卖这个小女,跟老婆打了一架。打架的原因不单纯是卖女心疼,而是老婆与婆婆过去积怨甚深,不愿为治婆婆的病卖掉自己的骨肉。卖了小女,娘的病也没治好,死在黄河边,软埋(没有棺材)在一个土窑里。走到洛阳,大女患天花,病死在慈善院里。扒火车去潼关,儿子没扒好,掉到火车轮下给轧死了。剩下老婆与他,来到陕西,给人拦地放羊。老婆嫌跟他生活苦,跟一个人拐子逃跑了。剩下他自己。麦秸垛前,他一把鼻涕一把泪地摊着手:\\

“我逃荒为个啥?我逃荒为图大家有个活命,谁知逃来逃去剩下我自己,我还逃荒干什么?早知这样,这荒不如不逃了,全家死还能死到一块,这死得七零八落的。”\\

这段话他儿子翻得很完全。我听了以后也感到是一个怪圈。我弄不明白的还有,现在不逃荒了,郭有运的新家有两层小楼,为什么还穿得这么破衣烂衫,仍像个逃荒的样子呢?如果不是老人家节俭的习惯,就是现实中的一切都不属于他。这个物质幸福的家庭,看来精神上并不愉快。这个家庭的家庭关系没有或永远没法理顺。我转过头对他儿子说:“老人家也不易,当年逃荒那个样子!”\\

谁知他儿子说:\\

“那怪他窝囊。要让我逃荒,我决不会那么逃!”\\

我吃了一惊:\\

“要让你逃,你怎么逃?”\\

他儿子:\\

“我根本不去陕西!”\\

我:\\

“你去哪儿?”\\

他儿子:\\

“我肯定下关东!关东不比陕西好过?”\\

我点头。关东肯定比陕西富庶,易于人活命。但我考察历史,我故乡没有向关东逃荒的习惯:闯关东是山东、河北人的事。我故乡遇灾遇难,流民路线皆是向西而不是往北。虽然西边也像他的故乡一样贫瘠。当然,一九四二、一九四三年还有一个特殊情况,就是东北三省已被日本人占了,去了是去当亡国奴。我把这后一条理由向他儿子谈了,谁知他一挥手上的“戈尔巴乔夫”,发出惊人论调:\\

“命都顾不住了,还管地方让谁占了?向西不当亡国奴,但他把你饿死了。换你,你是当亡国奴好呢,还是让饿死呢?不当亡国奴,不也没人疼你爱你管你吗?”\\

我默然,一笑。他提出的问题我解答不了。我想这是蒋委员长的失算,及他一九四九年逃到台湾的深刻原因。假如我处在一九四二年,我是找不管不闻不理不疼不爱我的委员长呢,还是找还能活命的东北关外呢?\\

告别郭有运和他的儿子,我又找到十李庄一位姓蔡的老婆婆。但这次采访更不顺利,还没等我与老婆婆说上话,就差点遭到她儿子的一顿毒打。姓蔡的婆婆今年七十岁,五十年前,也就二十岁。在随爹娘与两个弟弟向西逃荒时,路上夜里睡觉,全家的包袱、细软、盘缠、粮食,全部被人席卷一空。醒后发现,全家人只好张着傻嘴大哭。再向西逃没有活咱。她的爹娘只好把她卖掉,保全两个弟弟。一开始以为卖给了人家,但人贩子将她领走,转手又倒卖给窑子,从此做了五年皮肉生涯。直到一九四八年,国共两党的军队交战,隆隆炮声中,她逃出妓院,逃回家乡,像郭有运老汉一样,她现在的家庭、儿子、女儿一大家人,都是重起炉灶另建立的。她五年的肮脏非人生活,一直埋藏在她自己和大家的心底,除非邻里吵架时,被别的街坊娘们重新抖落一遍。但到了八十年代后期,她的这段生活,突然又显示出它特有的价值。本地的、外地的一些写畅销书的人,都觉得她这五年历史有特殊的现实意义,纷纷来采访她,要以她五年接客的种种情形,写出一本“我的妓女生涯”的自传体畅销书。从这题目看,畅销是必然的。为多写字的来采访,一开始使这个家庭很兴奋,原来母亲的经历还有价值,值得这些衣着干净人的关心。大家甚至感到很荣耀。但时间一长,当儿女们意识到写字的关心他们的目的,并不是为了关心他们自身,而是为了拿母亲的肮脏经历去为自己赚钱,于是她的儿女们,这些普普通通的庄稼人,突然感到自己受了骗,受了污辱。于是对再来采访的人,就怒目而视。为此,他们洋洋自得仍兴奋地沉浸在当年情形中的母亲,受到了她的儿子们严厉斥责。母亲从此对五十年前的事情又守口如瓶:已经说过的,也断然反悔。这使已经写下许多文字的人很尴尬。“我的妓女生涯”也因此夭折。这桩公案已经过去好几年了,现在我到这里来,又被她的儿子认为是来拿他母亲的肮脏经历赚钱的,要把已经夭折的“妓女生涯”再搭救起来。因此,我还没能与老婆婆说上话,他儿子的大棒,已差点落到我的头上。我不是一个多么勇敢的人,只好知难而退。而且我认为为了写这篇文章,去到处揭别人伤疤,特别是一个老女人肮脏的脓疮时,确实不怎么体面。我回去告诉了在乡派出所当副所长的我的小学同学,没想到他不这么认为,他怪我只是方式不对。他甩了甩手里的皮带说:\\

“这事你本来就应该找我!”\\

我:\\

“怎么,你对这人的经历很清楚?”\\

他:\\

“我倒也不清楚,但你要清楚什么,我把她提来审一下不就完了?”\\

我吃一惊,忙摆手:\\

“不采访也罢,用不着大动干戈。再说,她也没犯罪,你怎么能说提审就提审!”\\

他瞪大眼珠:\\

“她是妓女,正归我打击,我怎么不可以提审?”\\

我摆手:\\

“就是妓女,也是五十年前,提审也该那时的国民党警察局提审,也轮不到五十年后的你!”\\

他还不服气:\\

“五十年前我也管得着,看我把她抓过来!”\\

我忙拦住他,用话岔开,半天,才将气呼呼的他劝下。离开他时,我想,同学毕竟是同学呀。\\

为了把这次大逃荒记述下去,我们只好再次借助于《时代》周刊记者白修德。文章写到这里,我已清楚地意识到,白修德,必将成为这篇文章的主角,这不是因为别的,是因为一九四二年的河南大灾荒,已经没有人关心。当时的领袖不关心,政府不关心,各级官员在倒卖粮食发灾难财,灾民自己在大批死去,没死的留下的五十年后的老灾民,也对当年处以漠然的态度。这时,唯有一个外国人,《时代》周刊记者白修德,倒在关心着这片饥荒的土地和三百万饿死的人。自已的事情,自己这样的态度,自己的事情让别人关心、同情,说起来让五十年后的我都感到脸红。当然,白修德最初的目的,也不是为了关心我们的民众,他是出于一个新闻记者的敏感,要在大灾荒里找些可写的东西。无非是在找新闻的时候,悲惨的现实打动了他,震憾了他,于是产生了一个正常人的同情心,正义感,要为之一呼。这就有了以后他与蒋介石的正面冲突。说也是呀,一个美国人可以见委员长,有几个中国人,可以见到自己的委员长呢?怕是连政府的部长,也得事先预约吧。我们这些无依无靠的灾民,像自己父母一样的各级官员我们依靠不得,只好依靠一个其它力量并不强大的外国记者了。特别是后来,这种依靠也起了作用,这让五十年后的我深受震动、目瞪口呆。\\

白修德在一本《探索历史》的书中,描述了他一九四三年二月的河南之行。同行者是英国《泰晤士报》记者哈里林·福尔曼。在这篇文字开头我曾说到,在他们到达郑州时,曾在我的家乡吃过一顿“他能吃过的最好的筵席之一”。他们当时的行走路线是;从重庆飞抵宝鸡,剩陇海线火车从宝鸡到西安,到黄河,到潼关,然后进入河南。为防日本人炮击,从潼关换乘手摇的巡道车,整整一天,到达洛阳。所走的正是难民逃难的反方向。到达河南后,骑马到郑州,然后由郑州搭乘邮车返回重庆。从这行走路线看,是走马观花,只是沿途看到一些情形。记下的,都是沿途随时的所见所闻。这些所见是零碎的,所谈的见解带有很大的个人见识性。何况中美国情不同,这种个人见解离实际事务所包含的真正意蕴,也许会有一段距离。但我们可以为开这些见识,进入他的所见,进入细节;他肉眼看到的路边事实,总是真实的。我们可以根据这些真实的事实,去自己见识一九四三年的河南灾民大逃荒。我试图将他这些零碎的见闻能归纳得条理一些。\\

\begin{quote}
	一、灾民的穿戴和携带。灾民逃出来时,穿的都是他们最好的衣服,中年妇女穿著红颜绿色的旧嫁衣,虽然衣服上已是污迹斑斑;带的是他们家中最有价值的东西,烧饭铁锅,铺盖,有的还有一座老式座钟。这证明灾民对自己的故乡已彻底失去信心,没有留恋,决心离开家乡热土,连时间──座钟都带走了。白修德与他的伙伴在潼关车站睡了一夜。他说,那里到处是尿臊味、屎臭味和人身上的臭味。为了御寒,许多人头上裹着毛巾,有的帽子把帽耳朵放下来。他们在这里的目的,是为了等待往西去的火车,虽然这种等待是十分盲目的。\\

二、逃荒方式。不外是扒火车和行走。扒火车很不安全。白修德说,他沿途见到许多血迹斑班的死者。一种是扒上了火车,因列车被日本人的炮弹炸毁而丧命;有的是扒上了车厢顶,因夜里手指冻僵,失去握力,自己从车厢顶摔下摔死的;还有的是火车没扒上,便被行走的火车轧死的。轧死还好些,惨的是那些轧上又没轧死的。白见到一个人躺在铁轨旁,还活着,不停地喊叫,他的小腿被轧断,腿骨像一段白色的玉米杆那样露在外面。他还见到一个臀部轧得血肉模糊还没死去的人。白修德说,流血并不使他难过,难过的是弄不明白这些景象究竟是怎么回事。这么无组织无纪律的迁徙,他们各级政府哪里去了?──这证明白修德太不了解中国国情了。扒不上火车或对火车失望的,便是依靠自己的双腿,无目的无意识地向西移动。白修德说,整整一天,沿着铁路线,“我见到的便是这些由单一的、一家一户所组成的成群结队一眼望不到头的行列。”这种成群结队是自发的、无组织的,只是因为饥荒和求生的欲望,才使他们自动地组成了灾民的行列。可以想象,他们的表情是漠然的,他们也不知道,前边等待他们的是什么。唯一留在心中的信心,便是他们自己心中对前方未来的希望。也许能好一些,也许熬过这一站就好了。这是中国人的哲学,这又是白修德所不能理解的。灾民的队伍在寒冷的气候中行走。不论到哪里,只要他们由于饥寒或筋疲力尽而倒下,他们就再也起不来了。独轮车装着他们的全部家当,当爹的推着,当娘的拉着,孩子们跟着。缠足的老年妇女蹒跚而行。有的当儿的背着他们的母亲。在路轨两旁艰难行走在行列中,没有人停顿下来。如果有孩子伏在他的父亲或母亲的尸体上痛哭,他们会不声不响地从他身旁走过。没有人敢收留这啼哭的孩子。\\

三、卖人情况。逃荒途中,逃荒者所带的不多的粮食很快就会被吃光。接着就吃树皮、杂草和干柴。白边走边看到,许多人在用刀子、镰刀和菜刀剥树皮。这些树据说都是由爱好树木的军阀吴佩孚栽种的。榆树剥皮后就会枯死。当树皮、杂草、干柴也没得吃时,人们开始卖儿卖女,由那些在家庭中处于支配地位的人,去卖那些在家庭中处于被支配地位的人。这时同情心、家属关系、习俗和道德都已荡然无存,人们唯一的想法是要吃饭,饥饿主宰了世界上的一切。九岁男孩卖四百元,四岁男孩卖两百元,姑娘卖到妓院,小伙子往往被抓丁。抓丁是小伙子所欢迎的,因为那里有饭吃。如我的花爪舅舅。\\

四、狗吃人情况。由于沿途死人过多,天气又冷,人饥饿无力气挖坑,大批尸体暴尸野外,这给饥饿的狗提供了食品。可以说,在一九四三年的河南灾区,狗比人舒服,这里是狗的世界。白修德亲眼看到,出洛阳往东,不到一个小时,有一具躺在雪地的女尸,女尸似乎还很年轻,野狗和飞鹰,正准备瓜分她的尸体。沿途有许许多多像灾民一样多的野狗,都逐渐恢复了狼的本性,它们吃得膘肥肉厚。野地里到处是尸体,为它们的生存与繁殖提供了食物场。有的尸体已被埋葬了,野狗还能从沙土堆里把尸体扒出来。狗可能还对尸体挑挑拣拣。挑那些年轻的、口嫩的、女性温柔的。有的尸体已被吃掉一半,有的脑袋上的头肉也被啃得一乾二净,只剩下一个骷髅。白将这种情况,拍了不少照片。这些照片,对日后的没被狗吃仍活着的灾民,倒是起了不小的作用。\\

五、人吃人情况。人也恢复了狼的本性。当世界上再无什么可吃的时候,人就像狗一样会去吃人。白说,在此之前,他从未看到过任何人为了吃肉而杀死另一个人,这次河南之行,使他大开眼界,从此相信人吃人在世界上确有其事。如果人肉是从死人身上取下的倒可以理解,反正狗吃是吃,人吃也是吃;但情况往往是活人吃活人,亲人吃亲人,人自我凶残到什么程度?白见到,一个母亲把她两岁的孩子煮吃了;一个父亲为了自己活命,把他两个孩子勒死然后将肉煮吃了。一个八岁的男孩,逃荒路上死了爹娘,碰到汤恩伯的部队,部队硬要一家农民收容弃儿。后来这个孩子不见了。经调查,在那家农户的茅屋旁边的大坛子里,发现了这孩子的骨头;骨头上的肉,被啃得干干净净。还有易子而食的,易妻而食的。──写到这里,我觉得这些人不去当土匪,不去合伙谋杀,不去组成三K党,不去成立恐怖组织,实在辜负了他们吃人吃亲人吃孩子的勇气。从这点出发,我对地主分子范克俭舅舅气愤叙述的一帮没有逃荒的灾民揭竿而起,占据他家小楼,招兵买马,整日杀猪宰羊的情形,感到由衷地欢欣和敬佩。一个不会揭竿而起只会在亲人间相互残食的民族,是没有任何希望的。虽然这些土匪,被人用沾油的高粱秆给烧死了。他们的领头人叫毋得安。这是民族的脊梁和希望。\\
\end{quote}